
%Formeln, die benötigt werden für die Auswertung: c_k=.. habe ich mit \ref{eqn:Wärmekapazität} bezeichnet
%und dann noch für c_v=... das ist dann \ref{eqn:cv}
%und c_gm_g=.. für \ref{eqn:cm}
\section{Auswertung}
\subsection{Wärmekapazität des Kalorimeters}
Zu Beginn des Versuchs wird die spezifische Wärmekapazität $c_gm_g$ des Kalorimeters bestimmt,
da diese zur Bestimmung der Wärmekapazitäten von Graphit, Aluminium und Zinn benötigt wird.
Um diese zu bestimmen, werden die Temperaturen $T_x$, $T_y$ und $T_m$ und die Massen $m_x$ und $m_y$ abgelesen.
\newline
\begin{table}
  \centering
  \caption{Messwerte für das Kalorimeter}
  \begin{tabular}{c c c c c}
    \toprule
    $m_x$\,[g] & $T_x$\,$[\si{\degreeCelsius}]$& $m_y\,$ [g] & $T_y$\,$[\si{\degreeCelsius}]$ & $T_m$\,$[\si{\degreeCelsius}]$\\
    \midrule
    363.61 & 21.6 & 249.32 & 98.0 & 48.4\\
    \bottomrule
    \label{fig:kalorimeter}
  \end{tabular}
\end{table}
\newline
Mit Formel \ref{eqn:cm},den Messwerten aus Tabelle \ref{fig:kalorimeter} und der spezifischen Wärmekapazität für
für Wasser mit $c_w= 4.18\,\frac{\text{J}}{\text{gK}}$ ergibt
sich für die spezifische Wärmekapazität des Kalorimeters ein Wert von
\begin{equation*}
  c_gm_g = 408.88 \frac{\text{J}}{\text{K}}.
\end{equation*}
\subsection{Graphit}
Für die Bestimmung der Wärmpekapazität von Graphit wurden vier Messungen durchgeführt,
da bei der ersten Messungen ein systematischer Fehler begangen wurde. Deshalb wird diese
bei den Berechnung nicht weiter berücksichtigt, da sonst die Fehler sehr groß werden.
\newline
\begin{table}
  \centering
  \caption{Messwerte für Graphit}
  \begin{tabular}{c c c c c c}
    \toprule
    $\text{Messung}$ & $T_w$ $[\si{\degreeCelsius}]$ & $T_g$$ [\si{\degreeCelsius}]$ & $T_m$ $[\si{\degreeCelsius}]$ &  $c_k[\frac{\text{J}}{\text{gK}}]$\\
    \midrule
     1 & 22.6 & 43.0 & 26.0 & 6.20\\
     2 & 25.1 & 75.2 & 27.2 & 1.36\\
     3 & 27.6 & 74.8 & 29.2 & 1.09\\
     4 & 29.3 & 68.9 & 30.6 & 1.05\\
    \bottomrule
    \label{fig:graphit}
  \end{tabular}
\end{table}
\newline
Die Masse des Gewichtes betrug dabei $m_g=101.28$\,g und die Masse des Wassers $m_w=652.73$\,g.
Mit den Messwerten aus Tabelle \ref{fig:graphit} und Formel \ref{eqn:Wärmekapazität}
lässt sich eine spezifische Wärmekapazität von
\begin{align*}
  c_{\text{Graphit}} = (1.16\pm 0.10)\, \frac{\text{J}}{\text{gK}}
\end{align*}
bestimmen.
Der Mittelwert und die Standardabweichung errechnen sich dabei durch
\begin{equation}
  \bar{x} = \frac{1}{n} \sum{x_n}
  \label{eqn:Mittelwert}
\end{equation}
und
\begin{equation}
\upsigma = \frac{1}{\sqrt{n}} \sqrt{\frac{\sum{(x_n - \bar{x})^2}}{n-1} }.
\label{eqn:Standardabweichung}
\end{equation}
Zur Berchnug der Molwärmen für Graphit werden die materialspezifischen Werte in
Formel \ref{eqn:cv} eingesetzt. Daraus folgt für die einzelnen Messung für Graphit
\begin{align*}
 C_{\text{Graphit,2}} = 16.23\, \frac{\text{J}}{\text{molK}} \\
 C_{\text{Graphit,3}} = 13.01\, \frac{\text{J}}{\text{molK}} \\
 C_{\text{Graphit,4}} = 12.59\, \frac{\text{J}}{\text{molK}}.
\end{align*}
und für die mit \ref{eqn:Mittelwert} und \ref{eqn:Standardabweichung} gemittelte
Molwärme für Graphit
\begin{align*}
  C_{\text{Graphit}} = (13.94\pm 1.15)\, \frac{\text{J}}{\text{molK}}.
\end{align*}
\subsection{Aluminium}
\begin{table}
  \centering
  \caption{Messwerte für Aluminium}
  \begin{tabular}{c c c c c c}
    \toprule
    $T_w$ $[\si{\degreeCelsius}]$ & $T_a$ $[\si{\degreeCelsius}]$ & $T_m$$[\si{\degreeCelsius}]$ &$c_k[\frac{\text{J}}{\text{gK}}]$ \\
    \midrule
     31.3 & 45.6 & 33.7 & 4.71\\
    \bottomrule
  \end{tabular}
  \label{fig:aluminium}
\end{table}
Die Masse des Gewichtes betrug $m_a=147.72$\,g und die Masse des Wassers $m_w=728.08$\,g.
Für die einmalige Messung der spezifische Wärmekapazität von Aluminium lässt sich mit Tabelle \ref{fig:aluminium}
und \ref{eqn:Wärmekapazität} ein Wert von
\begin{align*}
  c_{\text{Aluminium}} = 4.71\, \frac{\text{J}}{\text{gK}}
\end{align*}
berechnen. Aus diesem folgt mit den Aluminium-spezifischen Eigenschaften und \ref{eqn:cv} eine Molwärme von
\begin{align*}
  C_{\text{Aluminium}} = 126.12 \, \frac{\text{J}}{\text{molK}}.
\end{align*}
\subsection{Zinn}
Wie bei den anderen Metallen wird auch bei dem schweren Metall Zinn zuerst die Wärmekapazität
und damit die Molwärme berechnet.
\begin{table}
  \centering
  \caption{Messwerte für Zinn}
  \begin{tabular}{c c c c c}
    \toprule
    $\text{Messung}$ & $T_w$ $[\si{\degreeCelsius}]$ & $T_z$ $[\si{\degreeCelsius}]$& $T_m$ $[\si{\degreeCelsius}]$ & $c_k[\frac{\text{J}}{\text{gK}}]$\\
    \midrule
     1 & 33.2 & 49.7 & 34.0 & 0.86\\
     2 & 33.9 & 54.7 & 34.6 & 0.59\\
     3 & 34.4 & 53.2 & 35.1 & 0.65\\
    \bottomrule
  \end{tabular}
  \label{fig:zinn}
\end{table}
\newline
Die Masse des Gewichts betrug dabei $m_z=214.50$\,g und die Masse des Wassers $m_w=769.81$\,g.
Mit dem Einsetzen der Werte aus Tabelle \ref{fig:zinn} in \ref{eqn:Wärmekapazität} ergibt sich
für die spezifische Wärmekapazität von Zinn ein Wert von
\begin{align*}
  c_{\text{Zinn}} = (0.70\pm 0.08)\, \frac{\text{J}}{\text{gK}}.
\end{align*}
Für die mehrfach gemessen Molwärmen von Zinn mit \ref{eqn:cv} folgt
\begin{align*}
C_{\text{Zinn,1}} = 100.46\, \frac{\text{J}}{\text{molK}} \\
C_{\text{Zinn,2}} = 68.08\, \frac{\text{J}}{\text{molK}} \\
C_{\text{Zinn,3}} = 75.80\, \frac{\text{J}}{\text{molK}}.
\end{align*}
Mittels \ref{eqn:Mittelwert} und \ref{eqn:Standardabweichung} folgt für die spezifische
Molwärme von Zinn
\begin{align*}
  C_{\text{Zinn}} = (81.45\pm 9.76)\, \frac{\text{J}}{\text{molK}}.
\end{align*}
\newpage
\section{Diskussion}
Für die Bestimmung der Molwärmen der Metalle Graphit, Aluminium und Zinn kam es beim
Abnehmen der Messwerte zu einigen Ungenauigkeiten. In der folgenden Tabelle sind diese
noch einmal aufgelistet:
\newline
\begin{table}
  \centering
  \caption{Vergleich der Messwerte mit dem Theoriewert}
  \label{fig:dis}
  \begin{tabular}{c c c}
    \toprule
    $\text{Metalle}$ & $\text{Molwärme}[\frac{\text{J}}{\text{molK}}]$ & $\text{Prozentuale Abweichung}[\%]$\\
    \midrule
     Graphit & 13.94\pm 1.15 & -44.11 \\
     Aluminium & 126.12 & 405.69 \\
     Zinn & 81.45\pm 9.76 & 226.58\\
    \midrule
    Theoriewert & 24.94\\
    \bottomrule
  \end{tabular}
\end{table}
\newline
Wie in Tabelle (\ref{fig:dis}) zu sehen ist, sind die Abweichungen von den gemessen Molwärmen zu den theoretischen
ziemlich groß. Lediglich die gemessene Molwärme für Graphit mit $13.94\,\frac{\text{J}}{\text{molK}}$ liegt in dem zu
erwartenen Bereich bis zu $25\,\frac{\text{J}}{\text{molK}}$. Durch die Molwärme kann geschlossen
werden, dass Graphit das leichteste gemessene Metall in diesem Versuch war.
\newline
Die bestimmten Molwärmen für Aluminium und Zinn sind allerdings viel größer als die Theoriewerte.
Dies lässt sich auf die Berechnung des $c_gm_g$-Wertes zurückführen. Mit den errechneten $408.88\,\frac{\text{J}}{\text{molK}}$
ist dieser Wert fast doppelt so groß wie der zu erwartende Wert von $200-300\,\frac{\text{J}}{\text{molK}}$.
\newline
Dieser starker Unterschied zwischen gemessen und theoretischen Werten lässt sich zu mindest zum Teil
durch die vielen Messungenauigkeiten erklären. Besonders die Temperaturen scheinen im Gegensatz
zu den Massen relativ ungenau zu sein. Das Abkühlen des Thermometers nach dem Messen dauerte sehr lange
und hatte beim Messen der Wassertemperatur noch vermutlich etwas Restwärme.
%evtl noch die Ungenauigkeit beim Messen der Temeratur der Proben wegen des temperaturfühlers.
\newline
Ein weiterer Punkt war die hohe Befestigung der Massen an dem Deckel, wodurch es
schwierig war das ganze Metall unter Wasser zu bringen, da es sonst überkochte. Weiterhin wurde im Laufe
der wiederholten Messung von Graphit und Zinn das Wasser nicht immer wieder aufgefüllt. Außerdem fand durch die fehlende
Isolation der Messkörper ein Wärmeaustausch mit der Luft statt.
\newline
Durch die sehr großen Abweichungen der Messwerte zu den Theoriewerten wird deutlich, dass die oszillatorischen
Bewegungen der Atome in Festkörpern quantenmechanisch betrachtet werden müssen, obwohl dabei nicht die
vielen Fehlerquellen während der Durchführung des Versuchs außer Acht gelassen werden dürfen.
