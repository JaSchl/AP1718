
  \section{Zielsetzung}
  In diesem Versuch soll herausgefunden werden ob Molwärmen mit klassischen Methoden
  bestimmt werden können oder ob dies nur quantenmechanisch möglich ist.
  Dafür wird das Doulong-Peptitsche Gesetz verwendet.

  \section{Theorie}
Die Molwärme $C$ beschreibt die Wärmemenge $dQ$, die erforderlich ist um ein Mol eines Stoffes um eine bestimmte
Temperatur $dT$ zu erwärmen.
Die Wärmekapazität bei konstantem Volumen kann mit der Formel
\begin{equation}
  C_v = \left(\frac{dQ}{dT}\right)_v
\end{equation}
beschrieben werden. Dabei entspricht $C = c\cdot m$.\\
Aus dem ersten Hauptsatz der Thermodynamik folgt:
\begin{equation*}
  C_v = \left(\frac{dU}{dT}\right)
\end{equation*}
Dabei ist $U$ die innere Energie eines Mols eines Stoffes.
Daraus folgt, dass die Molwärme bei konstantem Volumen immer $3R$ ist.
Das ist die Hauptaussage des Dulong-Peptitschen Gesetz.
Das Gesetz gilt für alle festen chemischen Elemente und für die meisten Stoffe schon ab
 Zimmertemperatur. Die ideale Gaskonstante beträgt $R = 8,314$ J/molK.\\
Die Formel für die Wärmekapazität bei konstantem Druck $C_p$ ist äquivalent.
Allerdings ist es schwieriger die Wärmekapazität bei konstantem Volumen  zu bestimmen,
da dieses sich bei Temperaturerhöhung ausdehnen kann und somit enorme Drücke aufgewendet werden müssten.
Um eine qualitative Aussage treffen zu können muss das Verhältnis zwischen $C_p$ und $C_v$ betrachtet werden.
\begin{equation}
  C_p - C_v = 9\alpha ^2\cdot \kappa\cdot V_0\cdot T
    \label{eqn:cv}
\end{equation}
Hierbei ist $\alpha$ der lineare Ausdehnngskoeffizient und $\kappa$ das Kompressionsmodul.\\
Die Wärmemenge, die der Probenkörper an das Wasser abgibt $Q_1$
muss gleich der Wärmemenge sein die das Wasser aufnimmt $Q_2$.
Also folgt mit
\begin{equation*}
  Q_1 = c_km_k(T_k -T_m)
\end{equation*}
und
\begin{equation*}
  Q_2 = (c_wm_w + c_gm_g)\cdot (T_m -T_w)
\end{equation*}
die spezifische Wärmekapazität für das Probematerial.
\begin{equation}
  c_k = \frac{(c_wm_w+c_gm_g)(T_m-T_w)}{m_k(T_k-T_m)}
  \label{eqn:Wärmekapazität}
\end{equation}
Alle Größen können gemessen werden.
Die Wärmekapazität für das Kalorimeter wird durch die Formel
\begin{equation}
  c_gm_g = \frac{c_wm_y(T_y -T'_m)-c_wm_x(T'_m - T_x)}{T'_m - T_x}
  \label{eqn:cm}
\end{equation}
bestimmt.
Dafür werden zwei Wassermengen ($m_x$ und $m_y$) mit zwei unterschiedlichen
Temperaturen ($T_x$ und $T_y$) vermischt und die Mischtemperatur ($T_m$) gemessen.

\section{Durchführung}
\subsection{Bestimmung der Wärmekapazität des Kalorimeters}

Zunächst muss die Wärmekapazität des Kalorimeters bestimmt werden.
Dafür werden zwei Bechergläser mit der Schnellwaage ausgemessen und mit Wasser gefüllt.
Das erste Becherglas wird wieder gemessen und der Inhalt in das Kalorimeter umgefüllt.
Das zweite Becherglas wird auf 100 °C erhitzt und danach auf der Schnellwaage gemessen.
Die Temperaturen im Kalorimeter und in dem Becherglas werden gemessen
und der Inhalt des Becherglases wird ebenfalls in das Kalorimeter gegeben.
ein Rührfisch sorgt dafür, dass die Temperaturen sich vermischen.
Nach einer Minute wird die Temperatur des Wassers erneut gemessen.
\subsection{Wärmekapazität unterschiedlicher Probenkörper}
Im zweiten Teil des Versuchs werden die Wärmekapazitäten für drei unterschiedliche Probenkörper bestimmt.
Es wurden Graphit, Aluminium und Zinn für die Messung verwendet.
Graphit und Zinn werden jeweils drei mal gemessen, die leichte Probe Aluminium wird nur einmal gemessen.
Die Proben sind mit einem Faden an einem Deckel befestigt,
sodass es möglich ist diese in ein Becherglas zu hängen.
Die Proben werden zunächst gewogen und das Gewicht von Faden und Deckel abgezogen.
Danach werden die Proben in einem mit Wasser gefülltem Becherglas gehängt und erhitzt,
bis das Wasser zu Kochen beginnt.
Nun wird die Temeratur des Wassers im Kalorimeter gemessen,
danach wird die Probe aus dem kochendem Wasser genommen und die Temeratur bestimmt.
Die Probe wird nun in das Kalorimeter gehängt.
Der Rührfisch sorgt für eine Vermischung der Temperaturen im Wasser.
Nach kurzer Zeit wird die neue Wassertemperatur gemessen.
