\documentclass[titlepage=firstiscover, bibliography=totoc, captions=tableheading]{scrartcl}
\author{Felix Geyer \\
  \texorpdfstring{\href{mailto:felix.geyer@tu-dortmund.de}{felix.geyer@tu-dortmund.de}\and}{,}
  Rune Dominik \\
  \texorpdfstring{\href{mailto:rune.dominik@tu-dortmund.de}{rune.dominik@tu-dortmund.de}}{}
  }
\title{V354: Gedämpfte und erzwungene Schwingungen}
\date{Durchführung: 10. Januar 2017 \\
      Abgabe: 17. Januar 2017}
\usepackage[aux]{rerunfilecheck}
\usepackage{polyglossia}
\setmainlanguage{german}
\usepackage{amsmath}
\usepackage{amssymb}
\usepackage{mathtools}
\usepackage{fontspec}

\usepackage{scrhack}
\usepackage{float}
\floatplacement{table}{htbp}
\floatplacement{figure}{htbp}

\usepackage[locale=DE, separate-uncertainty=true, per-mode=symbol-or-fraction, decimalsymbol=.]{siunitx}
%\usepackage{siunitx}

\usepackage[style=alphabetic]{biblatex}
\addbibresource{lit.bib}

\usepackage[section, below]{placeins}
\usepackage[labelfont=bf,
font=small,
width=0.9\textwidth,
format=plain,
indention=1em]{caption}
\usepackage{graphicx}
\usepackage{grffile}
\usepackage{subcaption}

\usepackage[math-style=ISO, bold-style=ISO, sans-style=italic, nabla=upright, partial=upright]{unicode-math}
\setmathfont{Latin Modern Math}

\usepackage[autostyle]{csquotes}

\usepackage[unicode]{hyperref}

\usepackage{bookmark}

\usepackage{booktabs}

\begin{document}

\section{mechanische Leistung des Kompressors}
Die mechanische Leistung wird mit Formel (13) bestimmt.
Aus der idealen Gasgleichung
\begin{equation*}
  p\cdot V = R\cdot m\cdot T
\end{equation*}
und
\begin{equation*}
  V =\frac{m}{\rho}
\end{equation*}
folgt für die Dichte:
\begin{equation*}
  \rho = \frac{\rho _0T_0p_a}{p_0T_2}\, .
\end{equation*}
Vorher gegeben sind:
\begin{align*}
  \rho _0 &= 5,51\cdot 10^{-3} \,\mathrm{\frac{kg}{L}}\\
  T &= 273,15 \, \mathrm{K} \\
  p &= 1\, \mathrm{Bar} = 10^4 \, \mathrm{Pa}\\
  \kappa &= 1,14
\end{align*}


Der Fehler berechnet sich mit der Formel
\begin{equation*}
  \Delta  N = \frac{1}{\kappa - 1} \left( p_b \sqrt[\kappa]{\frac{p_a}{p_b}} - p_a \right)
  \frac{1}{\rho}\cdot \Delta \left(\frac{dm}{dt}\right)\, .
\end{equation*}
Die Ergebnisse sind in Tabelle (\ref{tab:mech}) zu finden.
\begin{table}
  \centering
  \caption{Mechanische Leistung}
  \label{tab:mech}
  \begin{tabular}{c c c c}
    \toprule $Zeit[s]$ & $\rho [kg/m^{3}]$ & $\sqrt{p_a/p_b}$ & $N_{mech}[W]$  \\
    \midrule
    120 & 15,43 & 0,42300 & (-4,0 \pm 0,4) \\
    300 & 16,23 & 0,37443 & (-4,0 \pm 0,5) \\
    600 & 16,71 & 0,35028 & (-3,3 \pm 0,6) \\
    1080& 17,17 & 0,27510 & (-2,4 \pm 1,1)  \\
    \bottomrule
  \end{tabular}
\end{table}
\end{document}
