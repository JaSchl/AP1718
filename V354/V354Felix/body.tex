\section{Aufbau und Durchführung}
\subsection{Abklingdauer $T_\text{ex}$ und Dämpfungsstand $R_\text{eff}$}
\begin{wrapfigure}{r}{0.4\textwidth}
  \centering
  \includegraphics[width=0.4\textwidth]{Schaltung_Daempfung.png}
  \caption{Versuchsaufbau \cite{anleitung}.}
  \label{fig:Daempfung}
\end{wrapfigure}
Um die Abklingdauer $T_\text{ex}$ und den effektiven Dämfungswiderstand $R_\text{eff}$ der Schwingungsamplitude
des RLC-Schwingkreises aus Abbildug \ref{fig:Daempfung} zu berechnen, wird der kleinere der beiden festen Widerstände $R_\text{1}$ verwendet.
An dem Impulsgenerator wird eine Rechteckspannung erzeugt und mit dem Oszilloskop der Bereich der gedämpften Schwingung angezeigt.
Die Amplitude der Schwingung sollte bis auf den Faktor 8 abgeklungen sein, bevor ein neuer Impuls erzeugt wird.
Anhand der Einhüllenden der Schwingung kann dann $T_\text{ex}$ nach Gleichung \eqref{eqn:abkling} und $R_\text{eff}$ nach \eqref{eqn:mu}
 bestimmt werden.
\bigskip
\subsection{Durchführung der Bestimmung von $R_\text{ap}$}
Der Widerstand $R_\text{ap}$, bei dem die Schwingung das Verhalten eines aperiodischen Grenzfall besitzt, wird mit dem gleichen Aufbau
wie in Abbildung \ref{fig:Daempfung} bestimmt. Hier wird jedoch der feste Widerstand $R_\text{1}$ gegen einen variablen Widerstand
$R_\text{5\si{\kilo\ohm}}$ von $0-5 \; \si{\kilo\ohm}$ ausgetauscht.
Der Widerstand wird zuerst auf sein Maximum gestellt, sodass sich ein Kriechfall des Schwingkreises einstellt. Er wird nun so lange verringert,
bis sich der aperiodische Grenzfall einstellt. Dieser wird erreicht, wenn kein Überschwingen zu beobachten ist. Dazu wird sich langsam
mit dem Widerstand dem Wert angenähert, an dem gerade kein Überschwingen der Frequenz zu sehen ist.
\par
\subsection{Frequenzabhängigkeit der Kondensatorspannung}
Da auch die Ausgangsspannung des Tastkopfes von der Frequenz abhängt, muss die Spannung $U$ über den Tastkopf gemessen werden.
Bei dieser Messung wird der größere der festen Widerstände $R_\text{2}$ verwendet. Die Sinusspannung wird am Generator variiert.
Zunächst wird der Spannungsbereich grob eingestellt, um die Resonanzfrequenz zu finden. Der Bereich um die Resonanzfrequenz wird in
$1000\,\si{\hertz}$-Schritten vermessen.

\subsection{Frequenzabhängigkeit der Phase}
\begin{wrapfigure}{r}{0.5\textwidth}
  %\vspace{-10pt}
  \centering
  \includegraphics[width=0.5\textwidth]{Schaltung_Frequenzabhaengigkeit_Phase.png}
  %\vspace{-5pt}
  \caption{Versuchsaufbau \cite{anleitung}.}
  %\vspace{-5pt}
  \label{fig:Phase}
\end{wrapfigure}
Über das Zweikanal-Oszilloskop wird sowohl die Schwingung des LRC-Schwingkreis als auch die Erregerschwingung angezeigt.
Ein Aufbau der benötigten Schaltung ist in Abbildung \ref{fig:Phase} gezeigt.
Es wird die Phasendifferenz der Schwingungen in Abhängigkeit der Frequenz gemessen. Im Oszilloskop werden beide Spannungsverläufe parallel
 angezeigt. Die vorherigen Werte werden erneut eingestellt und der Abstand der Maxima bzw. Minima vermessen.
