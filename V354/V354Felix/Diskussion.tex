\section{Diskussion}

Der erste Versuchsteil zur Messung der Abklingdauer und des Dämpfungswiderstandes liefert gute Ergebnisse im Rahmen der Messungenauigkeit.
Die Einhüllende ist klar erkennbar und hat einen logischen Verlauf. \par\bigskip
Die extrem hohe Abweichung von $R_\text{eff}$ vom eingebauten Widerstand $R_\text{1}$ von $173.38 \, \si{\ohm}$ ist zum Einen durch den
Generatorinnenwiderstand von $50 \, \si{\ohm}$ zu erklären. Die restliche Abweichung ist zu hoch, um sie durch sonstige Widerstände zu erklären.
\par\bigskip

Die Abweichung von $R_\text{ap}$ ist durch zusätzliche Widerstände im Versuchsaufbau und vor allem durch Ablesefehler bedingt.
Da der zu beobachtende Bereich sehr klein gewesen ist, war es nicht ohne weiteres möglich, ein eventuelles Unterschwingen durch den Nullpunkt,
oder ein stärkeres Relaxationsverhalten der Spannungskurve zu unterscheiden.
\par\bigskip

Die Berechnung der Güte weist eine sehr große Abweichung zum Theoriewert auf, diese ist jedoch nicht auf unbeachtete, weitere Innenwiderstände
zurückzuführen. Hier liegt wahrscheinlich ein systematischer Fehler vor. An den Grenzen zu $\nu\to\infty$ und $\nu\to 0$ liefert der Graph
den erwarteten Verlauf.
\par\bigskip

Zur Auswertung der Frequenzabhängigkeit der Phase ist anzumerken, dass der Verlauf der Phase wie angenommen im Bereich $\varphi = \frac{\pi}{4}$
bis $\varphi = \frac{3\pi}{4}$ seine Resonanzfrequenz besitzt.
