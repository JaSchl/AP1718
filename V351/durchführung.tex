
\section{Durchführung}
\subsection{Vorbereitung}
Zur Vorbereitung auf den Versuch werden die Fourier-Koeffizienten einer
Rechteck-, Dreieck- und Sägezahnspannung bestimmt. Dafür wird die Annahme
gemacht, dass die Funktion entweder gerade oder ungerade sind.
\subsubsection{Rechteckspannung}
Für die Bestimmung der Rechteckspannung wird eine ungerade Funktion definiert, aus der sich
die Koeffizenten:
\begin{align*}
 a_{0} = 0,\ \text{da}\ f(t)\ \text{ungerade}\\
 a_{n} = 0,\ \text{da}\ f(t)\ \text{ungerade}\\
 b_{n} =	\begin{cases}
			\dfrac{2}{n\pi} ,\ \text{für}\ n\ \text{ungerade}\\ \\
      0 ,\ \text{für}\ n\ \text{gerade}\\\\
		\end{cases}
\end{align*}
berechnen lassen.

\subsubsection{Dreieckspannung}
Die Fourrierkoeffizienten einer geraden Funktion berechnen sich zu:
\begin{align*}
  a_{n} =  \begin{cases}
 					\dfrac{-4}{(n\pi)^{2}} ,\ \text{für}\ n\ \text{ungerade}\\ \\
 					0 ,\ \text{für}\ n\ \text{gerade}\\\\
 				\end{cases}\\
  b_{n} = 0,\ \text{da}\ f(t)\ \text{gerade}\\
\end{align*}

\subsubsection{Sägezahnspannung}
Zur Berechnung einer Sägezahnspannung wird wie bei der Rechteckspannung eine
ungerade Funktion verwendet. Daraus folgt für die Amplituden der Oberschwingung:
\begin{align*}
 a_{0} = 0,\ \text{da}\ f(t)\ \text{ungerade}\\
 a_{n} = 0,\ \text{da}\ f(t)\ \text{ungerade}\\
 b_{n} = (-1)^{n+1} \dfrac{2T}{n\pi}.
\end{align*}
\subsection{Fourier-Analyse}
Der Versuchsaufbau der Fourier-Analyse besteht aus einem Funktionsgenerator,
der als Signalquelle dient, und einem Digitaloszilloskop. Vom Funktionsgenerator
werden jeweils Rechteck-, Dreieck- und Sägezahnspannugen erzeugt. Durch eine
Fast-Fourier-Transformation sind die generierten Spannungen auf dem Oszilloskop
zu sehen. Anhand dessen können die Frequenzen und die Amplituden der Peaks
abgelesen werden.

\subsection{Fourier-Synthese}
Für den zweiten Teil des Versuchs wird zusätzlich ein Oberwellengenerator eingeschaltet.
Bei der Fourier-Synthese werden mit dem Oberwellengenerator bis zu neun Sinus-Schwingungen
mit ganzzahligem Frquenzverhältnis eingespeist.
Im Gegensatz zur Fourier-Analyse bleibt die Phase bei diesem Teil des Experiments
allerdings konstant. Zu Beginn werden die Phasen anhand der Lissajous-Figuren des
Phasenverhältnis der Grundschwingung und der jeweiligen Überschwingung eingestellt
und während des gesamten Versuchs nicht verändert. Durch das schrittweise Einschalten
der einzelnen Oberwellen ist die Summenschwingung am Oszilloskop zu sehen.
