

\section{Durchführung}
\subsection{Vorbereitung}
Zur Vorbereitung auf den Versuch werden die Fourier-Koeffizienten einer
Rechteck-, Dreieck- und Sägezahnspannung bestimmt. Zur Vereinfachung wurden die
Funktionen als gerade beziehunsweise ungerade angenommen.
\subsubsection{Rechteckspannung}
Für die Bestimmung der Rechteckspannung wird eine ungerade Funktion definiert, aus der sich
die Koeffizenten:
\begin{align*}
 a_{0} = 0,\ \text{da}\ f(t)\ \text{ungerade}\\
 a_{n} = 0,\ \text{da}\ f(t)\ \text{ungerade}\\
 b_{n} =	\begin{cases}
			\dfrac{1}{n\pi} ,\ \text{für}\ n\ \text{ungerade}\\ \\
      0 ,\ \text{für}\ n\ \text{gerade}\\\\
		\end{cases}
\end{align*}
berechnen lassen.

\subsubsection{Dreieckspannung}
Die Fourrierkoeffizienten einer geraden Funktion berechnen sich zu:
\begin{align*}
  a_{n} =  \begin{cases}
 					\dfrac{1}{(n\pi)^{2}} ,\ \text{für}\ n\ \text{ungerade}\\ \\
 					0 ,\ \text{für}\ n\ \text{gerade}\\\\
 				\end{cases}\\
  b_{n} = 0,\ \text{da}\ f(t)\ \text{gerade}\\
\end{align*}

\subsubsection{Sägezahnspannung}
Zur Berechnung einer Sägezahnspannung wird wie bei der Rechteckspannung eine
ungerade Funktion verwendet. Daraus folgt für die Amplituden der Oberschwingung:
\begin{align*}
 a_{0} = 0,\ \text{da}\ f(t)\ \text{ungerade}\\
 a_{n} = 0,\ \text{da}\ f(t)\ \text{ungerade}\\
 b_{n} = (-1)^{n+1} \dfrac{1}{n\pi}.
\end{align*}
\subsection{Fourieranalyse}
Der Versuchsaufbau der Fourieraalyse besteht aus einem Funktionsgenerator,
der als Signalquelle dient und einem Digitalszilloskop, welches die Fourieranalyse
durchführt.
\newline
Die Abtastrate am Oszilloskop muss größer als das Doppelte der maximalen
Oberschwingungsfrequenz eingestellt werden, sodass gilt
\begin{align}
  v_a > 2v\ua{max}.
\end{align}

Vom Funktionsgenerator werden jeweils Rechteck-, Dreieck- und Sägezahnspannugen erzeugt.
Anhand dessen können die Frequenzen und die Amplituden der Peaks mittels der
Cursorfunktion abgelesen werden.

\subsection{Fouriersynthese}
Zur Fouriersynthese wird anschließend ein Oberwellengenerator verwendet.
Zuerst werden die Phasenverhältnisse zwischen Grund- und Oberschwingung eingestellt.
Dafür wird das Oszilloskop auf den X-Y-Betrieb umgestellt. Auf dem X-Eingang wird die
Grundschwingung eingespeist und auf dem Y-Eingang eine ganzzahlige Oberwelle.

Auf dem Oszilloskop sind sogenannte Lissajousfiguren erkennbar, welche bei Überlagerung
zweier harmonischer, rechtwinklig zueinander stehenden Schwingungen entstehen.
Die Phasendifferenz zwischen Grund- und Oberschwingung soll auf $\Updelta\varphi = 0$
eingestellt werden.

Durch das schrittweise Aufsummieren der einzelnen Oberwellen ist die Summenschwingung
am Oszilloskop zu sehen.
