\section{Zielsetzung}
Ziel dieses Versuchs ist der Umgang mit den Fourier-Koeffizienten.
\section{Theorie}
Die Beschreibung vieler periodischer Funktionen erfolgt durch das Fouriersche Theorem:
\begin{equation}
f(t) = \frac{a_0}{2} + \sum_{n=1}^\infty (a_n cos(n\frac{2\pi}{T}t) + b_n sin(n\frac{2\pi}{T}t)).
\end{equation}
Die Koeffizienten $a_n$ und $b_n$ beschreiben die Amplituden der Oberschwingungen und können durch
\begin{equation}
a_n = \frac{T}{2}\int_{0}^{T} f(t) \cos(n\frac{2\pi}{T}t) dt
\end{equation}
und
\begin{equation}
b_n = \frac{T}{2}\int_{0}^{T} f(t) \sin(n\frac{2\pi}{T}t) dt
\end{equation}
berechnet werden.
Um das Rechnen mit dem Theorem zu erleichtern, bietet es sich an, eine gerade oder ein ungerade
Funktion zu wählen. Für eine gerade Funktion sind die Koeffizienten $b_n$ gleich 0, und für eine ungerade Funktion
die Koeffizienten $a_n$.
Dabei ist es unerlässlich, dass die Reihe gleichmäßig konvergent und somit stetig an jeder Stelle ist.
Ist die Funktion an einer Stelle unstetig, wird von dem Gibbschen Phänomen gesprochen. Dieses beschreibt den Fall,
dass die Funtkion nicht mehr durch die Fourierreihe genähert werden kann, was eine endliche Abweichung zur Folge hat.

Mithilfe der Fourier-Transformation kann das gesamte Frequenzspektrum einer zeitabhänigen Funktion
bestimmt werden, welche nicht periodisch sein muss:
\begin{equation}
g(v) = \int_{-\infty}^{\infty} f(t) e^{ivt} dt.
\end{equation}
