
\section{Theorie}
Sich wiederholende Vorgänge, welche nach einer Periodendauer $T$ oder einer Distanz $D$
erneut ihren Anfängswert annehmen, werden als periodische Vorgänge bezeichnet.
Sie lassen sich durch
\begin{align}
  f(t+T) &= f(t)
\end{align}
beziehungsweise
\begin{align}
  f(x+D) &= f(x)
\end{align}
beschreiben. Die häufigsten periodischen Funktionen sind die Sinus- und Cosinusfunktionen.
Diese sind $2\pi$ periodische und im Wertebereich von -1 und 1 definiert.
Mit ihrer Amplitude $a$ oder $b$ und der Periodendauer $T$ können sie als
\begin{align}
 f(t) &= a\cos(n\frac{2\pi}{T}t)
\end{align}
beziehungsweise
\begin{align}
 f(t) &= b\sin(n\frac{2\pi}{T}t)
\end{align}
dargestellt werden. Fast alle anderen stetige, periodischen Vorgänge der Natur können
durch sie beschrieben werden. Für diese gilt das Fouriersche Theorem
\begin{equation}
f(t) = \frac{a_0}{2} + \sum_{n=1}^\infty (a_n cos(n\frac{2\pi}{T}t) + b_n sin(n\frac{2\pi}{T}t)),
\end{equation}
wenn sie stetig sind, die Reihe also gleichmäßig konvergiert.
Die Koeffizienten $a_n$ und $b_n$ beschreiben die Amplituden der Oberschwingungen und können durch
\begin{equation}
a_n = \frac{T}{2}\int_{0}^{T} f(t) \cos(n\frac{2\pi}{T}t) dt
\end{equation}
und
\begin{equation}
b_n = \frac{T}{2}\int_{0}^{T} f(t) \sin(n\frac{2\pi}{T}t) dt
\end{equation}
berechnet werden.
In der Fourier-Entwicklung treten lediglich Vielfache der Grundvfrequenz $v = \frac{1}{T}$ auf,
welche auch als harmonische Oberschwingungen bezeichnet werden. 0, $\frac{\pi}{2}$, $\pi$ und $2\pi$
können dabei nur als Phasen vorkommen.
\newline

Zur Bestimmung von $a_n$ und $b_n$ wird eine Fourier-Analyse durchgeführt. Je nach Funktion
können die Koeffizienten wegfallen. Für gerade Funktionen gilt $a_n = 0$ und für ungerade
$b_n = 0$.

Auch nicht stetige Funktionen lassen sich durch eine Fourierreihe nähern. Allerdings tritt
dann an der Unstetigkeitsstelle $t_0$ eine endliche Abweichung der Reihe von der Funktion auf.
Dies wird als Gibbsches Phänomen bezeichnet.

Mit Hilfe einer Fourier-Transformation kann das gesamte Frequenzspektrum einer zeitabhänigigen
Funktion bestimmt werden:
\begin{equation}
g(v) = \int_{-\infty}^{\infty} f(t) e^{ivt} dt.
\end{equation}
Die Funktion $g(v)$ stellt dabei das Frequenzspektrum der Funktion $f$ dar, welche selbst für
nicht-periodische Funktionen gilt.
