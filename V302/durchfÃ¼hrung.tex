
  \section{Durchführung}
  Zunächst wird die Speisespannung abgelesen, indem der Frequenzregler mit dem Oszillographen verbunden wird.
  \subsection{Wheatonesche Brückenschaltung}
  Die Wheatstonesche Brücke wird wie in Abbildung (\ref{fig:whe}) gezeigt aufgebaut und mit dem Oszillographen verbunden.
  Es werden zwei unterschiedliche unbekannte Widerstände gewählt, die bestimmt werden sollen.
  Dies geschieht für drei unterschiedliche Werte für $R_2$.
  $R_3$ und $R_4$ werden bestimmt, indem das Potentiometer auf einen Widerstand gestellt wird,
  sodass die Schwingung auf dem Oszillographen minimal ist.

  \subsection{Kapazitätsmessbrücke}
  Für die Kapazitätsmessbrücke wird der Widerstand aus dem ersten Teil mit einem Kondensator erweitert und ein regelbarer Widerstand geschaltet,
  sodass die Schaltung aus Abbildug (\ref{fig:kapa}) entsteht.
  Der Messvorgang aus dem ersten Teil wird  wiederholt.
  Hier werden allerdings auch die bekannten Kondensatoren dreimal ausgetauscht.
  Nach der Messung werden die Widerstände entfernt, sodass nur eine unbekannte Kapazität bestimmt werden muss.
  Daraufhin wird die Messung wiederholt.
  \subsection{Induktivitätsmessbrücke}
  Der Aufbau aus Abbildung (\ref{fig:indu}) wird geschaltet.
  Der unbekannt Widerstand und die Spule werden bestimmt,
  indem mit dem regelbarem Widerstand und dem Potentiometer ein minimum gefunden wird.
  Die Messung wird nur für eine unbekannte Spule durchgeführt.
  \subsection{Maxwell-Brücke}
  Die Maxwell-Brücke wird wie in Abbildung (\ref{fig:max}) geschaltet.
  Der unbekannte Wert der Spule aus dem vorherigen Messungsteil wird wieder verwendet um ihn zu bestimmen.
  Dies geschieht für drei unterschiedliche Widerstände und Kapazitäten.

  \subsection{Wien-Robinson-Brücke}
  Nun wird die Schaltung aus Abbildung (\ref{fig:wien}) aufgebaut.
  Die Frequenz wird hier erstmalig verändert.
  Bei 0,2 kHz anfangend wird sie hochgedreht um ein Minium herauszufinden.
  Dafür wird die Frequenz gegen die Spannung aufgetragen.
