\section{Diskussion}
In diesem Vesuch wurden drei verschiedene Methoden verwendet, um die Zeitkonstante RC zu bestimmen.
Die Messergebnisse sind in der folgenden Tabelle nochmal aufgelistet:
\begin{table}
  \caption{Bestimmte Werte für RC bei den verschiedenen Messungen}
  \label{tab:Zusammenfassung}
  \begin{tabular}{c | c c c}
    \toprule  $Messung$ & $3.1$ & $3.2$ & $3.3$\\
    \midrule  $RC \, \pm \, \increment RC$ & $0.000775 \, \pm \, 0.000070$ &
              $0.000786 \, \pm \, 0.000013$ & $0.00502 \, \pm \, 0.00025$\\
    \bottomrule
  \end{tabular}
\end{table}
Wie in der Tabelle zu sehen ist, weichen die Messwerte zum Teil sehr stark voneineander ab.
Die Zeitkonstanten bei der Bestimmung über den Thermodruck und die Amplitude befinden sich jedoch
in der selben Größenordnung.
Das Ergebnis für die Zeitkonstante mithilfe der Bestimmung über die Phasenverschiebung besitzt allerdings
eine sehr große Abweichung zu den ersten beiden Messung. Hierbei war es besonders schwierig den zeitlichen Abstand der beiden
Nulldurchgänge der beiden Schwnigungen und deren Schwingungdauer am Oszillographen genau abzulesen.
Diese Messungenauigkeiten genau wie die vernachlässigten Innenwiderstände könnten diese doch sehr große Abweichung erklären.
