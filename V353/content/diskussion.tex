\section{Diskussion}
\label{sec:Diskussion}
Die aus den verschiedenen Messmethoden bestimmten Werte der Zeitkonstante $RC$ sind in Tabelle \ref{tab:diskussion} zusammengefasst.
\begin{table}
  \centering
  \caption{Ergebnisse des Versuchs.}
  \label{tab:diskussion}
  \begin{tabular}{c S}
    Messung & {$RC$ in s} \\
    \midrule
    Entladevorgang & $\SI{0.300 \pm 0.016}{ms}$ \\
    Frequenzabhängigkeit der Spannungsamplitude & $\SI{0.0593 \pm 0.0005}{s}$ \\
    Frequenzabhängigkeit der Phase & $\SI{0.068 \pm 0.011}{s}$ \\
  \end{tabular}
\end{table}

Es zeigt sich, dass die Ergebnisse der zweiten und dritten Messung sehr ähnlich sind, das Ergebnis des Entladevorgangs jedoch stark davon abweicht.
Dies könnte damit zusammenhängen, dass die Frequenzabhängigkeiten beide in der gleichen Messung aufgenommen wurden. So kann in dieser Messung ein
systematischer Fehler nicht allein durch die ähnlichen Ergebnisse für RC ausgeschlossen werden.
Weiterhin wurden für die Berechnung der Ausgleichsgeraden einige Werte manuell aus dem Oszilloskopbild abgelesen, wohingegen die Werte für die anderen
beiden Messungen direkt vom Oszilloskop angezeigt wurden. Dies, sowie eine zu geringe Anzahl an abgelesenen Werten kann zu weiteren Ungenauigkeiten
geführt haben. Zudem fällt auf, dass die Messwerte zum Teil deutlich von der Ausgleichsgeraden abweichen.

Bei der Frequenzabhängigkeit der Kondensatorspannung hingegen liegen die Messwerte fast alle direkt auf der Ausgleichsfunktion und der Fehler ist mit
$\SI{0,84}{\%}$ sehr gering. Bei der Frequenzabhängigkeit der Phasenverschiebung sind die Abweichungen in den hohen Frequenzbereichen sehr hoch,
dennoch stimmt der Wert für $RC$ relativ gut mit dem aus der zweiten Messung überein. Die hohen Abweichungen in den oberen Frequenzbereichen könnten
daher kommen, dass die Differenz der Nulldurchgänge $a$ am Oszilloskop nicht direkt abgelesen werden konnte, sondern mithilfe von Cursorn, die per Hand
auf die Nulldurchgänge eingestellt werden mussten und dies wurde mit steigender Frequenz immer ungenauer und schwieriger.

Im Polarplot des Zusammenhangs zwischen Kondensatorspannung und Phasenverschiebung sind die Messwerte im Großen und Ganzen auch sehr nah an der
Theoriekurve, nur im höheren Bereich sind zwei Werte mit etwas größerer Abweichung, die sich ebenfalls durch die oben genannte Ungenauigkeit bei
der Bestimmung der Phasenverschiebung erklären lässt.
