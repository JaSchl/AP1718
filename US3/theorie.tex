
\section{Zielsetzung}
Mi Hilfe dieses Versuches sollen charakteristische Eigenschaften von Strömungen untersucht werden,
wie sie beispielsweise bei Blutströmungen in Gefäßen und im Herzen vorkommen.
\section{Theorie}
Der Dopplereffekt beschreibt die Frequenzänderung, die sich ergibt wenn eine Schall-Quelle und ein Objekt sich
relativ zueinander bewegen.
Bewegt sich die Schall-Quelle auf das Objekt zu, wird die Frequenz $\nu_{kl}$ des empfangenen Schalls höher.
Entfernt sich die Quelle, sinkt die Frequenz $\nu_{gr}$.
Die Frequenz kann mit
\begin{equation}
  \nu_{kl/gr} = \frac{\nu_0}{1 \mp \frac{v}{c}}
\end{equation}
beschrieben werden.
Ist die Quelle in Ruhe und das Objekt bewegt sich, ergibt sich die Formel
\begin{equation}
  \nu_{h/n} = \nu_0 \left(1\pm\frac{v}{c}\right)
\end{equation}
für hohe Frequenzen $\nu_h$ und niedrige Frequenzen $\nu_n$.
c ist die Schallgeschwindigkeit und v die Geschwindigkeit des Objekts.\\
Trifft eine Ultraschallwelle auf ein bewegtes Objekt ergibt sich wie oben beschrieben eine Frequenzverschiebung.
Die Frequenzverschiebung kann mit Formel
\begin{equation}
  \Delta \nu = \nu_0 \frac{v}{c}\cdot(\cos\alpha+\cos\beta)
\end{equation}
beschrieben werden.
$\alpha$ und $\beta$ sind die Winkel zwischen der Geschwindigkeit und der ein- bzw. auslaufenden Welle.
$\alpha$ und $\beta$ sind bei dem verwendeten Impuls-Echo-Verfahren identisch, sodass sich für die Frequenzverschiebung
\begin{equation}
  \Delta\nu = 2\nu_0\frac{v}{c}\cos{\alpha}
  \label{eqn:2}
\end{equation}
ergibt.\\
Ultraschall kann unteranderem mit dem reziporken piezo-elektrischen Effekt erzeugt werden.
Dabei wird ein Piezokristall als Sender und Empfänger verwendet.
Der Kristall wird in einem elektrischen Wechselfeld zu Schwingungen angeregt, was zur Abstrahlung von Ultraschallwellen führt.


\section{Durchführung}
Zunächst wird der Computer gestartet und eine 2\,MHz Sonde angeschlossen.
Außerdem wird das Programm "FlowView" gestartet.\\
Der Versuchsaufbau besteht zum einen aus einem Schlauch, in das eine Flüssigkeit gefüllt ist.
Die Flüssigkeit besteht aus einem Gemisch aus Wasser, Glycerin und Glaskugeln.
Zum anderen ist ein Gerät befestigt, mit dem die Geschwindigkeit der Flüssigkeit eingestellt werden kann.\\
Es gibt drei Rohre mit den Durchmessern 7\,mm, 10\,mm und 16\,mm.
Auf diese Rohre werden Doppler-Prismen aufgesetzt,
die ermöglichen die Ultraschallsonde in drei verschiedenen Einschallwinkeln anzusetzen.
Die Winkel betragen 15°, 30° und 60°.\\
Im ersten Teil des Versuchs wird zunächst für fünf unterschiedliche Geschwindigkeiten die Frequenz aufgenommen.
Das wird für alle drei unterschiedlich dicken Rohre und unterschiedlichen Winkel durchgeführt.\\
Im zweiten Teil des Versuchs wird nur das Rohr mit der Dicke 10\,mm betrachtet.
Mit dem Dopplerwinkel von 15° wird das Strömungsprofil bestimmt indem die Strömungsgeschwindigkeiten
und der Streuintensitätswert für unterschiedliche Tiefen aufgenommen wird.
