\documentclass[
  bibliography=totoc,     % Literatur im Inhaltsverzeichnis
  captions=tableheading,  % TabellenÃŒberschriften
  titlepage=firstiscover, % Titelseite ist Deckblatt
]{scrartcl}

% Paket float verbessern
\usepackage{scrhack}

% Warnung, falls nochmal kompiliert werden muss
\usepackage[aux]{rerunfilecheck}

% unverzichtbare Mathe-Befehle
\usepackage{amsmath}
% viele Mathe-Symbole
\usepackage{amssymb}
% Erweiterungen fÃŒr amsmath
\usepackage{mathtools}

% Fonteinstellungen
\usepackage{fontspec}
% Latin Modern Fonts werden automatisch geladen
% Alternativ zum Beispiel:
%\setromanfont{Libertinus Serif}
%\setsansfont{Libertinus Sans}
%\setmonofont{Libertinus Mono}

% Wenn man andere Schriftarten gesetzt hat,
% sollte man das Seiten-Layout neu berechnen lassen
\recalctypearea{}

% deutsche Spracheinstellungen
\usepackage{polyglossia}
\setmainlanguage{german}


\usepackage[
  math-style=ISO,    % ┐
  bold-style=ISO,    % │
  sans-style=italic, % │ ISO-Standard folgen
  nabla=upright,     % │
  partial=upright,   % ┘
  warnings-off={           % ┐
    mathtools-colon,       % │ unnötige Warnungen ausschalten
    mathtools-overbracket, % │
  },                       % ┘
]{unicode-math}

% traditionelle Fonts fÃŒr Mathematik
\setmathfont{Latin Modern Math}
% Alternativ zum Beispiel:
%\setmathfont{Libertinus Math}

\setmathfont{XITS Math}[range={scr, bfscr}]
\setmathfont{XITS Math}[range={cal, bfcal}, StylisticSet=1]

% Zahlen und Einheiten
\usepackage[
  locale=DE,                   % deutsche Einstellungen
  separate-uncertainty=true,   % immer Fehler mit \pm
  per-mode=symbol-or-fraction, % / in inline math, fraction in display math
]{siunitx}

% chemische Formeln
\usepackage[
  version=4,
  math-greek=default, % ┐ mit unicode-math zusammenarbeiten
  text-greek=default, % ┘
]{mhchem}

% richtige AnfÃŒhrungszeichen
\usepackage[autostyle]{csquotes}

% schöne BrÌche im Text
\usepackage{xfrac}

% Standardplatzierung fÃŒr Floats einstellen
\usepackage{float}
\floatplacement{figure}{htbp}
\floatplacement{table}{htbp}

% Floats innerhalb einer Section halten
\usepackage[
  section, % Floats innerhalb der Section halten
  below,   % unterhalb der Section aber auf der selben Seite ist ok
]{placeins}

% Seite drehen fÃŒr breite Tabellen: landscape Umgebung
\usepackage{pdflscape}

% Captions schöner machen.
\usepackage[
  labelfont=bf,        % Tabelle x: Abbildung y: ist jetzt fett
  font=small,          % Schrift etwas kleiner als Dokument
  width=0.9\textwidth, % maximale Breite einer Caption schmaler
]{caption}
% subfigure, subtable, subref
\usepackage{subcaption}

% Grafiken können eingebunden werden
\usepackage{graphicx}
% größere Variation von Dateinamen möglich
\usepackage{grffile}

% schöne Tabellen
\usepackage{booktabs}

% Verbesserungen am Schriftbild
\usepackage{microtype}

% Literaturverzeichnis
\usepackage[
  backend=biber,
]{biblatex}
% Quellendatenbank
\addbibresource{lit.bib}
\addbibresource{programme.bib}

% Hyperlinks im Dokument
\usepackage[
  unicode,        % Unicode in PDF-Attributen erlauben
  pdfusetitle,    % Titel, Autoren und Datum als PDF-Attribute
  pdfcreator={},  % ┐ PDF-Attribute sÀubern
  pdfproducer={}, % ┘
]{hyperref}
% erweiterte Bookmarks im PDF
\usepackage{bookmark}

% Trennung von Wörtern mit Strichen
\usepackage[shortcuts]{extdash}

\author{%
  AUTOR A\\%
  \href{mailto:authorA@udo.edu}{authorA@udo.edu}%
  \texorpdfstring{\and}{,}%
  AUTOR B\\%
  \href{mailto:authorB@udo.edu}{authorB@udo.edu}%
}
\publishers{TU Dortmund – FakultÀt Physik}

\begin{document}

\begin{table}
  \centering
  \caption{Rechteckiger Stab, einseitige Einspannung}
  \label{tab:data2}
  \begin{tabular}{c c c  }
    \toprule $x \, \,  in \,\, m$ & $D(x) \,\, in \,\,  m$ & $Lx^2-\frac{1}{3}x^3$\\
    \midrule
    0.03 & 0.00780 & 0.0005346\\
    0.05 & 0.00744 & 0.0014683\\
    0.10 & 0.00641 & 0.0057067\\
    0.15 & 0.00525 & 0.0124650\\
    0.20 & 0.00491 & 0.0214933\\
    0.25 & 0.00254 & 0.0325417\\
    0.30 & 0.00120 & 0.0453600\\
    0.35 & 0.00088 & 0.0596983\\
    \bottomrule
  \end{tabular}
\end{table}

\begin{table}
  \centering
  \caption{Zylindrischer Stab, einseitige Einspannung}
  \label{tab:data2}
  \begin{tabular}{c c c  }
    \toprule $x \, \,  in \,\, m$ & $D(x) \,\, in \,\,  m$ & $Lx^2-\frac{1}{3}x^3$ \\
    \midrule
    0.03 & 0.00800 & 0.0005328\\
    0.05 & 0.00757 & 0.0014633\\
    0.10 & 0.00640 & 0.0056867\\
    0.15 & 0.00506 & 0.0124200\\
    0.20 & 0.00342 & 0.0214133\\
    0.25 & 0.00151 & 0.0324167\\
    0.30 & 0.00080 & 0.0451800\\
    \bottomrule
  \end{tabular}
\end{table}

\begin{table}
  \centering
  \caption{Zylindrischer Stab, beidseitige Einspannung}
  \label{tab:data2}
  \begin{tabular}{c c c  }
    \toprule $x \, \,  in \,\, m$ & $D(x) \,\, in \,\,  m$ & $3L^2x-4x^3$ \\
    \midrule
    0.03 & 0.00842 & 0.032508\\
    0.05 & 0.00836 & 0.053861\\
    0.10 & 0.00816 & 0.104721\\
    0.15 & 0.00804 & 0.149582\\
    0.20 & 0.00790 & 0.185443\\
    0.25 & 0.00775 & 0.209303\\
    0.30 & 0.00684 & 0.218164\\
    0.35 & 0.00700 & 0.020902\\
    0.40 & 0.00716 & 0.178885\\
    0.45 & 0.00735 & 0.124745\\
    0.50 & 0.00765 & 0.043606\\
    0.55 & 0.00800 & -0.067533\\
    \bottomrule
  \end{tabular}
\end{table}
\end{document}
