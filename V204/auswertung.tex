
  \section{Auswertung}
\subsection{Statische Methode}
\subsubsection{Temperaturverläufe}

Die Temperaturverläufe von dem breiteren ($T_1$)  und dem dünneren ($T_4$) Messingstab sind in Abbildung (2) zu sehen.
Der Verlauf der beiden Graphen ist sehr ähnlich.
Die Temperatr des dickeren Stabes steigt leicht geringer als die Temeratur der dünneren Probe.
Nach ca. $18,75\, s$ steigen die Graphen mit konstanter Steigung.
An diesem Punkt haben beide Proben eine Temperatur von $24,07\, °C$.
Nach ca.  $120,31\, s$ steigen die Temeraturkurven langsamer an und nähren sich einem Maximum.
Zu diesem Zeitpunkt besitzt der dickere Stab eine Temperatur von ca. $42,96\, °C$
und der dünnere eine Temperatur von ca. $43,33\, °C$.

Die Temperaturverläufe von Aluminium ($T_5$) und Edelstahl ($T_8$) sind in Abbildung (3) zu sehen.
Der Temperaturverlauf ähnelt dem Verlauf von dem zuvor betrachteten Messing.
Der Graph beginnt allerdings schon nach $9,38\, s$ stark zu steigen.
Zu diesem Zeitpunkt hat die Probe eine Temperatur von $24,12\, °C$.
Nach $118,75 \, s$ besitzt die Probe eine Temperatur von $56,47\, °C$.
Die Edelstahl-Probe besitzt zu diesem Zeitpunt eine Temperatur von $26,47\, °C$.
Die Temperaturkurve dieser Probe wächst sehr langsam und sehr flach.
Sie ist zu den anderen Dreien durch ihren Verlauf am unterschiedlichsten.
Es fällt auf, dass nach Beendigung der Hitzezugabe die Temperaturkurve ca $130\, s$ weiter steigt.

\begin{table}
  \centering
  \caption{Maxima}
  \label{tab:hp}
  \begin{tabular}{c c c }
    \toprule $\symup{Material}$ & $T[°C]$ & $t[s]$ \\
    \midrule
    Messing          & 43,70 & 135,94 \\
    $\symup{Messing_{schmal}}$ & 43,70 & 129,69 \\
    Aluminium        & 56,47 & 118,75 \\
    Edelstahl        & 30,59 & 229,69 \\
    \bottomrule
  \end{tabular}
\end{table}
In Tabelle (\ref{tab:hp}) sind die Maxima der erzeugten Graphen aufgetragen.
Dabei wurde die höchst erreichte Temperatur der Probe während der Messung aufgetragen.
Es stellt sich heraus, dass Aluminium die beste und Edelstahl die schlechteste Wärmeleitfähigkeit besitzt.

\subsubsection{Wärmestrom}
Der Wärmestrom wird für Messing und Edelstahl
%Für Aluminium ist dies nicht möglich, da $\partial T$ nicht mehr bestimmt werden kann.
 mit Formel (\ref{eqn:kappa})berechnet.\
Für die Wärmeleitfähigkeit $\kappa$ werden folgende Werte eingesetzt:
\begin{align*}
  \kappa_{Messing} &= 120\, \mathrm{W/mK} \\
  \kappa_{Edelstahl} &= 15\, \mathrm{W/mK}
\end{align*} \cite{on2}

Die Querschnittsfläche des Stabes beträgt $4,8 \cdot 10^{-5}\, m^2$ \cite[3]{on1}
und der Abstand zwischen den beiden Thermoelementen beträgt $4\, cm$
Die Werte für $\partial T$ werden aus den Abbildungen (4) und (5) entnommen
und sind in Tabelle (\ref{tab:dq}) zu finden.\\
Die Ergebnisse für den Wärmestrom sind ebenfalls in Tabelle (\ref{tab:dq}) zu finden.
\begin{table}[H]
  \centering
  \caption{Wärmestrom}
  \label{tab:dq}
  \sisetup{table-format=2.1}
\begin{tabular}{S[table-format=3.1] S S S S }
    \toprule
    & \multicolumn{2}{c}{Messing} & \multicolumn{2}{c}{Edelstahl} \\
\cmidrule(lr){2-3}\cmidrule(lr){4-5}
{$t[s]$} & {$(T_2-T_1)[K]$} & {$dQ/dt[W]$} & {$(T_8-T_7)[K]$} & {$dQ/dt[W]$} \\
    \midrule
    32,25 & 282,41  & -40,67  & 279,44 & -5,03 \\
    50,00 & 284,08  & -40,90  & 284,58 & -5,12 \\
    56,13 & 284,82  & -41,01  & 288,86 & -5,19\\
    82,25 & 285,19  & -41,07  & 291,54 & -5,25 \\
    100,00& 285,19  & -41,07  & 293,51 & -5,28 \\
    \bottomrule
  \end{tabular}
\end{table}


\subsubsection{Temperaturdifferenz}

Die Temperaturdifferenz zwischen den beiden Thermoelementen des Messingstabes sind in Abbildung (4) zu sehen.
$\symup T_2$ ist die Temperatur die näher am Peltierelement liegt.
Die Temperaturkurve steigt erst sehr stark und dann immer langsamer.
Das Temperaturmaximum liegt bei ca. $11,85\, °C$.
Dieser Wert wird nach ca. $109,68\, s$ errwicht.
Ab diesem Punkt fällt die Kurve rapide ab.
Die Temperaturdifferenz der beiden Thermoelemente von Edelstahl ist in Abbildung (5) zu sehen.
Das Maximum liegt bei $120\, s$und hat einen Wert von $22,86\, °C$.
Die Temeratur steigt sehr gleichmäßig bis zu dem Punkt, an dem der Abkühlvorgang beginnt.
Im Vergleich zu den Temperaturverläufen aus Abbildung (2) und (3), ist hier auch zu sehen,
dass Edelstahl die schlechteste Wärmeleitfähigkeit der vier getesteten besitzt.


\subsection{Dynamische Methode}
Die Amplituden und die Phasendifferenz $\Delta t$,
die in den Tabellen (\ref{tab:mes}), (\ref{tab:al}) und (\ref{tab:ed}) zu finden sind,
wurden mithilfe der Abbildungen (6), (7) und (8) bestimmt.
Die Wärmeleitfähigkeit $\kappa$ wurde mit Formel (\ref{eqn:kappa2}) bestimmt.
\begin{table}[H]
  \centering
  \caption{Messingstab}
  \label{tab:mes}
  \begin{tabular}{c c c c}
    \toprule $A_{1_{fern}}[°C]$ & $A_{2_{nah}}[°C]$ & $\Delta t[s]$ & $\kappa[W/mK]$ \\
    \midrule
    1,5 & 4,5  & 14,286 & 167,411 \\
    1,5 & 4,25 & 9,524  & 264,564 \\
    1,5 & 4,5  & 14,286 & 167,411 \\
    1,5 & 4,5  & 9,524  & 250,799 \\
    1,5 & 4,5  & 9,524  & 250,799 \\
    1,5 & 4,25 & 14,286 & 176,599 \\
    1,5 & 4,25 & 9,524  & 264,564 \\
    1,5 & 4,25 & 9,524  & 264,564 \\
    1,5 & 4,5  & 9,524  & 250,799 \\
    \bottomrule
  \end{tabular}
\end{table}
Daraus folgt nun der Mittelwert für die Wärmeleitfähigkeit:
\begin{equation}
\overline \kappa = (228,612\pm 31,616)\, \mathrm{\frac{W}{mK}}
\end{equation}
Der Literaturwert liegt bei $120\, W/mK$ \cite{on2}\\
Somit ergibt sich eine Abweichung von $90,51\, \%$
\begin{table}[H]
  \centering
  \caption{Aluminiumstab}
  \label{tab:al}
  \begin{tabular}{c c c c}
    \toprule $A_{5_{fern}}[°C]$ & $A_{6_{nah}}[°C]$ & $\Delta t[s]$ & $\kappa$ \\
    \midrule
    3,261  & 6,739 & 9,524  & 268,933 \\
    3,043  & 6,422 & 9,524  & 261,368 \\
    3,043  & 6,304 & 9,524  & 268,023 \\
    3,043  & 6,304 & 9,524  & 268,023 \\
    3,043  & 6,304 & 9,524  & 268,023 \\
    3,043  & 6,304 & 9,524  & 268,023 \\
    2,826  & 6,087 & 9,524  & 254,908 \\
    3,043  & 6,087 & 9,524  & 282,167 \\
    2,826  & 6,087 & 9,524  & 254,908 \\
    \bottomrule
  \end{tabular}
\end{table}
Der Mittelwert ergibt sich zu:
\begin{equation}
\overline \kappa = (263,013\pm 8,361)\, \mathrm{\frac{W}{mK}}
\end{equation}
Der Wert, für die Leitfähigkeit von Aluminium liegt bei $221\, W/mK$ \cite[275]{b1}\\
Somit liegt die Abweichung zum Literaturwert bei $19,01\, \% $
\begin{table}[H]
  \centering
  \caption{Edelstahlstab}
  \label{tab:ed}
  \begin{tabular}{c c c c}
    \toprule $A_{7_{nah}}[°C]$ & $A_{8_{fern}}[°C]$ & $\Delta t[s]$ & $\kappa$ \\
    \midrule
    9,038 & 1,154 & 50   & 24,876 \\
    8,077 & 1,154 & 52,7  & 24,965 \\
    \bottomrule
  \end{tabular}
\end{table}
Der Mittelwert für die Wärmeleitfähigkeit beträgt
\begin{equation}
\overline \kappa = (24,931\pm 0,046)\, \mathrm{\frac{W}{mK}}.
\end{equation}
Der zu erwartende Literaturwert liegt bei $15\, W/mK$. \cite{on2}\\
Die prozentuelle Abweichung beträgt $66,21\, \%$.


\section{Diskussion}
Im ersten Teil wurde der Fehler gemacht die Messung nicht zu beenden, nachdem T7 eine Temperatur von $45\, °C$ angenommen hat.
Dies führte dazu, dass das Abkühlen ebenfalls auf dem Graphen abgebildet ist
und im Nachhinein nicht mehr genau nachvollziehbar ist, bis zu welchem Punkt erwärmt wurde.\\

Während des Erwärmens, wurde von einer idealen Isolierung ausgegangen, die so nicht umsetzbar ist,
sodass davon ausgegangen werden kann, dass bei den Temperaturverläufen eine Ungenauigkeit auftritt.

Die Bestimmung des Wärmestroms von Aluminium war nicht möglich, da keine Messwerte vorhanden sind
und die Differenz zwischen $T_5$ und $T_6$ in keiner Graphik abgebildet ist.

Da die komplette Auswertung auf den Graphen, die während des Versuchs gedruckt wurden
basiert, sind alle angegebenen Temperaturen und Zeiten fehlerbehaftet.
Die Amplituden und die Phasendifferenz wurden ebenfalls mithilfe der Graphen bestimmt.
Die Ungenauigkeit dieser Methode fällt besonders bei der Phasendifferenz auf.
Es nur möglich die Differenz auf einen Millimeter genau zu bestimmen.
Dieser Millimeter macht allerdings einen Unterschied von $4,762\, s$.
Die Genauigkeit für die Amplitudenmessung liegt zwischen $0,192\, °C$ und $0,25\, °C$ auf je ein Millimeter.\\

Die Bestimmung des Wärmestroms und der Vergleich der berechneten Wärmeleitfähigkeiten mit der Literatur,
basiert auf den im Literaturverzeichnis angegebenen Quellen.
Beim herausuchen dieser Quellen fiel auf,
dass es mehrere unterschiedliche Angaben zu den einzelnen Wärmeleitfähigkeiten gab.
Somit sind diese Werte nicht als feste Zahl zu betrachten, sondern als eine Nährung.
Besonders bei Edelstahl wurden Werte in einer größeren Spannweite gefunden,
da dieser stark von der Legierung abhängig ist.

Zusammenfassend kann gesagt werden, dass der Versuch zu dem Ergebnis führt,
dass Edelstahl die schlechteste und Aluminium die beste Leitfähigkeit besitzt.
