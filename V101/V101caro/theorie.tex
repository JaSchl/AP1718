\documentclass[scrartcl]
\usepackage{scrhack}
\usepackage{amsmath}
\usepackage{amssymb}
\usepackage[
  math-style=ISO,    % ┐
  bold-style=ISO,    % │
  sans-style=italic, % │ ISO-Standard folgen
  nabla=upright,     % │
  partial=upright,   % ┘
  warnings-off={           % ┐
    mathtools-colon,       % │ unnötige Warnungen ausschalten
    mathtools-overbracket, % │
  },                       % ┘
]{unicode-math}

\begin{document}
\section{Theorie}
Als Trägheitsmoment wird der Widerstand eines Körpers gegen die Änderung seiner Rotation beschrieben,
welches abhängig von der Masse $M$ und dem Radius $r$ ist.
Die rotierenden Punktmasse besitzt ein Trägheitsmoment von $I = mr^2$.
Durch das Aufsummieren der einzelnen Trägheitsmomente wird das Gesamtträgheitsmoment berechnet.
\begin{align}
 I= \sum_{i = 0}^n r_i^2\cdot m_i
 \end{align}
 Entsprechend gilt für eine kontinuierliche Massenverteilung
 \begin{equation}
   \label{eqn:Trägheitsmoment}
   I = \int r^2\mathup{dm}
 \end{equation}}
Zur Berechnung der Trägheitsmomente bekannter Symmetrien wurden folgenede Formeln verwendet:
\emph{Kugel}:
\begin{align}
  I_K = \frac{2}{5}mR^2
\end{align}
\emph{Zylinder}:
\begin{align}
  I_{Z} &= \frac{mR^2}{2}
  I_{Zh} = m(\frac{R^2}{4} + \frac{h^2}{12})
\end{align}
Wichtig zu beachten ist, dass dabei die Rotationsachse durch den Schwerpunkt des Körpers verlöuft.
Ist dies nicht der Fall beschreibt der \textit{Satz von Steiner} die parallele Verschiebung
der Rotationsache, wobei $I_s$ als das Trägheitsmoment bezüglich der Schwerpunktsachse
und $a$ als der Abstand der Drehachse zur Schwerpunktsachse definiert ist.
\begin{equation}
  \label{eqn:Steiner}
  I = I_s +m\cdot a^2
\end{equation}}
\subsection{Bestimmung der Trägheitsmomente}
Für das Drehmoment gilt
\begin{equation}
  \vec{M} = \vec{F} \times \vec{r}
\end{equation}
In einem schwingungsfähigen System hängen Trägheitsmoment und Schwingungsdauer zusammen.
Dieser Zusammenhang wird durch
\begin{equation}
  T = 2\pi \sqrt{\frac{I}{D}}
\end{equation}
beschrieben, wobei $I$ für das Trägheitsmoment des Körpers und $D$ für die
Winkelrichtgröße steht.
Für kleine Winkel lässt sich der Betrag des Drehmoments als
\begin{equation}
  M = D\cdot\varphi
\end{equation}
mit dem Auslenkwenkel $\varphi$ darstellen.
\end{document}
