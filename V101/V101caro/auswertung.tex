\section{Auswertung}
\subsection{Bestimmung der Winkelrichtgröße und des Eigenträgheitmomentes der Drillachse}
Für die Bestimmung der Winkelrichtgröße wurde folgende Formel verwendet:
\begin{equation}
  D = \frac{\lvert F \rvert \cdot \lvert r \rvert}{\varphi}
\end{equation}

Unsere Messergebenisse der Winkel und der jeweilig wirkenden Kraft sind in Tabelle 1 eingetragen.

\begin{table}
  \centering
  \caption{Wirkende Kraft in N in Abhängigkeit des Drehwinkels bei verschiedenen Abständen.}
  \label{tab:data1}
  \begin{tabular}{c c c c c}
    \toprule Abstand in m & $\frac{1}{3} \pi$ & $\frac{2}{3} \pi$ & $ \pi$ & $\frac{4}{3} \pi$ \\
    \midrule
    0.10 & 0.31 & 0.61 & 0.89  & 1.20 \\
    0.15 & 0.20 & 0.40 & 0.595 & 0.89 \\
    0.20 & 0.16 & 0.30 & 0.44  & 0.59 \\
    \bottomrule
  \end{tabular}
\end{table}

Mithilfe der Messdaten werden die jeweiligen Winkelrichtgrößen berechnet und gemittelt,
sodass sich folgender Wert für die "statische" Winkelrichtgröße ergibt:

\begin{equation}
  D_{stat} = (0.0291 \pm 0.0003) \symup{Nm}
\end{equation}

Die Winkelrichtgröße kann auch dynamisch bestimmt werden. Die Messergebnisse dazu befinden sich
in Tabelle 2. Zur Bestimmung von $D_{dyn}$ wird in einem Graphen $T^2$ gegen $a^2$ aufgetragen,
wobei $a$ dem jeweiligen Abstand zum Drillachsenmittelpunkt $\lvert r \rvert$ entspricht und $T$
der jeweilig gemessenen Schwingungsdauer. Mittels linearer Regression wird dann die Steigung $m$ sowie
der y-Achsenabschnitt $b$ berechnet. Mithilfe dieser Werte wird dann ebenfalls die
Winkelrichtgröße mit folgender Formel berechnet:

\begin{equation}
  D_{dyn} = \frac{8\pi^2 \cdot m_{zyl}}{m}
\end{equation}

Der Fehler ergibt sich dabei mithilfe Gauß´scher Fehlerrechnung aus folgender Formel:

\begin{equation}
  \sigma_{D_{dyn}} = \frac{8 \pi^2 \cdot m_{zyl}}{m^2} \cdot \sigma_m
\end{equation}

\begin{table}
  \centering
  \caption{Schwingungsdauern bei jeweiligen Abständen.}
  \label{tab:data2}
  \begin{tabular}{c c c c }
    \toprule $a \, \,  in \,\, m$ & $a^2 \,\, in \,\,  m^2$ & $T \,\, in \,\, s$ & $T^2 \,\, in  \,\, s^2$ \\
    \midrule
    0.0500 & 0.0025 & 2.79867 &  7.8325\\
    0.1000 & 0.0100 & 3.61067 &  13.037\\
    0.1245 & 0.0155 & 4.12933 &  17.051\\
    0.1504 & 0.0226 & 4.71533 &  22.234\\
    0.1751 & 0.0306 & 5.29200 &  28.005\\
    0.2000 & 0.0400 & 5.84000 &  34.106\\
    0.2261 & 0.0511 & 6.49133 &  42.137\\
    0.2505 & 0.0628 & 7.10800 &  50.524\\
    0.2740 & 0.0751 & 7.69600 &  59.228\\
    0.3000 & 0.0900 & 8.33733 &  69.511\\
    \bottomrule
  \end{tabular}
\end{table}

\begin{figure}
  \includegraphics[width=\textwidth]{plot1.pdf}
  \caption{Die Quadrate der Schwingungsdauer gegenüber den Abstandsquadraten}
\end{figure}

Die Regression wird dabei von Python durchgeführt, indem an folgende Funktion gefittet wird.
Die Parameter $C_1$ und $C_2$ entsprechen dabei der Steigung und dem y-Achsenabschnitt:

\begin{align}
  T^2 &= C_1 * a^2 + C_2 \\
  C_1 &= m \\
  C_2 &= b
\end{align}

Die Steigung der Regressionsgerade beträgt $m = (706 \pm 2) \frac{s^2}{m^2}$ und
der y-Achsenabschnitt hat einen Wert von $b = (6.12 \pm 0.09) s^2$. So ergibt sich für
die "dynamische" Winkelrichtgröße folgender Wert:

\begin{equation}
  D_{dyn} = (0.02925 \pm 0.00008) \symup{Nm}
\end{equation}

Zu sehen ist, dass $D_{dyn}$ im Fehlerbereich von $D_{stat}$ liegt, dies aber anders herum nicht
gilt. Da $D_{dyn}$ den geringeren Fehler hat, wird im folgenden mit diesem Wert weitergerechnet.

Mithilfe von $D_{dyn}$ und des y-Achsenabschnittes von Abbildung 1 können nun mit folgenden Formeln
auch das Trägheitsmoment der Drillachse und der Fehler bestimmt werden:

\begin{align}
  I_D        &= I_{ges} - 2 \cdot I_{Z} - I_{Stange}\\
             &= \frac{b \cdot D_{dyn}}{4 \pi^2} - 2 \cdot I_{Z} - I_{Stange}\\
  \sigma_I   &= \sqrt{ \left( \frac{D_{dyn}}{4\pi^2} \cdot \sigma_b \right) ^2 + \left( \frac{b}{4\pi^2} \cdot \sigma_{D_{dyn}} \right) ^2}
\end{align}

Die Trägheitsmomente von Zylinder und Stange müssen vorher noch berechnet werden und hinterher abgezogen werden.
Dabei ergeben sich für $I_{Z}$ und $I_{Stange}$ folgende Werte:

\begin{align}
  I_Z        &= m_{zylinder} \cdot \left( \frac{R_Z^2}{4} + \frac{H_Z^2}{12} \right)\\
  I_{Stange} &= \frac{1}{12} \cdot m_{Stange} \cdot H_{Stange}^2
\end{align}

Bei $R_Z$ und $H_Z$ handelt es sich dabei um Radius und Höhe des Zylinders. Da wir in unserer Formel
den y-Achsenabschnitt verwenden, muss der Steinersche Satz \eqref{eqn:Steiner} nicht beachtet werden, da wir sozusagen
mit einer Schwingungsdauer bei $r=0$ rechnen.
Mit dem obigen y-Achsenabschnitt ergibt sich dann für das Trägheitsmoment der Drillachse folgender Wert:

\begin{equation}
  I_D = (0.0045 \pm 0.0007) \symup{kg m^2}
\end{equation}

\subsection{Bestimmung der Trägheitsmomente zwei verschiedener Körper}
Im folgenden Abschnitt wird das Trägheitsmoment einer homogenen Kugel sowie einer
homogenen Zylinders sowohl theoretisch über Abmessung und Masse sowie experimentell über
die Schwingungsdauer bestimmt und danach verglichen.

\subsubsection{Trägheitsmoment einer homogenen Kugel}
Vor dem Start der Messungen wurden zuerst sowohl Masse als auch Durchmesser der Kugel ermittelt.
Dabei wird beides als fehlerfrei angenommen:

\begin{align}
  r_K &= 0.06883 \symup{m} \\
  m_{Kugel} &= 0.33821 \symup{kg}
\end{align}

Mit diesen Werten lässt sich das "theoretische" Trägheitsmoment wie folgt bestimmen:

\begin{equation}
  I_{Kugel,theo} = \frac{2}{5} m_{Kugel} \cdot r_K^2 = 0.00154 \,\, \symup{kg m^2}
\end{equation}

Als nächstes wird das Trägheitsmoment über die Schwingungsdauer bestimmt. Die gemessenen Werte
sind in Tabelle 3 eingetragen.

\begin{table}
  \centering
  \caption{Schwingungsdauern der homogenen Kugel}
  \begin{tabular}{c c c c c c c c c c c}
    \toprule $Messung$ & $1$ & $2$ & $3$ & $4$ & $5$ & $6$ & $7$ & $8$ & $9$ & $10$ \\
    \midrule $Dauer [s]$ & 7.50 & 7.50 & 7.47 & 7.44 & 7.50 & 7.41 & 7.50 & 7.47 & 7.56 & 7.47\\
    \bottomrule
  \end{tabular}
\end{table}

Für die Schwingungsdauer werden die Werte noch durch 5 geteilt und gemittelt. Wir erhalten somit einen Wert von :

\begin{equation}
  T_{Kugel} = (1.4964 \pm 0.0026) \symup{s}
\end{equation}

Damit ergibt sich für das Trägheitsmoment folgender Wert:

\begin{equation}
  I_{Kugel,exp} = \frac{T^2 D_{dyn}}{4\pi^2} - I_D = (0.00154 \pm 0.00007) \symup{kg m^2}
\end{equation}

Der Fehler errechnet sich wie folgt mithilfe der Gauß´schen Fehlerfortpflanzung:

\begin{equation}
  \sigma_I = \sqrt{(2T \frac{D_{dyn}}{4\pi^2} \cdot \sigma_T)^2 + (\frac{T^2}{4\pi^2} \cdot \sigma_{D_{dyn}})^2}
    \label{eqn:sigma}
\end{equation}

Verwendet wird dabei die dynamisch bestimmte Winkelrichtgröße, da dieser Größe der selbige
Messprozess zugrunde liegt, den wir auch bei der Bestimmung der Trägheitsmomente über die
Schwingungsdauer verwendet haben.

\subsubsection{Trägheitsmoment eines homogenen Zylinders}

Die Messwerte für die Schwingungsdauern finden sich in Tabelle 4. Die Rechnungen sind dabei analog
zu denen der homogenen Kugel.

\begin{align}
  r_Z &= 0.04012  \symup{m} \\
  h_Z &= 0.13990  \symup{m} \\
  m_Z &= 1.00580  \symup{kg}
\end{align}

\begin{equation}
  I_{Zylinder,theo} = \frac{1}{2} m_Z \cdot r_Z^2 = 0.000809 \,\, \symup{kg m^2}
\end{equation}

\begin{table}
  \centering
  \caption{Schwingungsdauern des homogenen Zylinder}
  \begin{tabular}{c c c c c c c c c c c}
    \toprule $Messung$ & $1$ & $2$ & $3$ & $4$ & $5$ & $6$ & $7$ & $8$ & $9$ & $10$ \\
    \midrule $Dauer [s]$ & 5.22 & 5.28 & 5.28 & 5.25 & 5.28 & 5.22 & 5.28 & 5.22 & 5.25 & 5.25\\
    \bottomrule
  \end{tabular}
\end{table}

Für die Schwingungsdauer erhalten wir somit einen Wert von:

\begin{equation}
  T_{Zylinder} = (1.0506 \pm 0.0017) \symup{s}
\end{equation}

Damit errechnet sich für das Trägheitsmoment und dessen Fehler \eqref{eqn:sigma} folgender Wert:

\begin{equation}
  I_{Zylinder,exp} = (0.00070 \pm 0.00007)  \symup{kg m^2}
\end{equation}

\newpage

\subsection{Das Trägheitsmoment einer Holzpuppe in zwei verschiedenen Positionen}

Als letzter Teil der Messungen soll das Trägheitsmoment eine Holzpuppe in zwei verschiedenen
Positionen bestimmt werden. In Position 1 sind die beiden Arme seitlich ausgestreckt und
bei Position 2 sind der rechte Arm sowie das rechte Bein nach vorne ausgestreckt.

%Hier noch Referenz für Sebastian

\subsubsection{Abschätzung des Trägheitsmomentes aus den Abmessungen}
In der folgenden Tabelle sind alle Abmessungen der Puppe eingetragen. Die Indizes entsprechen
dabei dem entsprechendem Körperteil [t = Torso, k = Kopf, a = Arm, b = Bein].

\begin{table}
  \centering
  \caption{Abmessungen der Holzfigur in Metern}
  \begin{tabular}{c c c c c c c c}
    \toprule $r_t$ & $h_t$ & $r_k$ & $h_k$ & $r_a$ & $h_a$ & $r_b$ & $h_b$ \\
    \midrule 0.023034 & 0.13114 & 0.015805 & 0.071 & 0.1002 & 0.17864 & 0.01136 & 0.20042 \\
    \bottomrule
  \end{tabular}
\end{table}

Bei den Rechnungen werden die einzelnen Körperteile als Zylinder angenommen. Da die Körperteile
allerdings teilweise erheblich von diesen Formen abweichen kann folgendes jedoch lediglich als
Abschätzung betrachtet werden. Somit macht auch eine Fehlerrechnung in diesem Fall keinen Sinn
und wird in diesem Abschnitt nicht durchgeführt.

Das Gesamtträgheitsmoment entspricht somit einer Summation der Trägheitsmomente der einzelnen
Körperteile. Um die Massen der einzelnen Körperteile zu ermitteln, wurde das Gesamtvolumen der
Figur berechnet und damit die Dichte bestimmt. Mit dem Volumen der einzelnen Körperteile wurden
dann die einzelnen Massen [Tabelle 6] bestimmt:

\begin{table}
  \centering
  \caption{Volumen und Massen der einzelnen Körperteile}
  \begin{tabular}{c | c c}
    \toprule $Objekt$ & $Volumen [cm^3]$ & $Masse [kg]$ \\
    \midrule
    Torso & 0.0004373 & 0.134536 \\
    Kopf  & 0.0001110 & 0.034293 \\
    Arm   & 0.0001125 & 0.034679 \\
    Bein  & 0.0001625 & 0.050011 \\
  \end{tabular}
\end{table}

\newpage

Für Position 1 gilt somit folgende Formel:

\begin{align}
  I_{ges} &= I_T + I_K + 2 \cdot I_A + 2 \cdot I_B - I_D\\
          &= \frac{1}{2} m_t r_t^2 + \frac{1}{2} m_k r_k^2 + 2 \cdot (m_a(\frac{r_a^2}{4} + \frac{h_a^2}{12}) + m_a(r_t + \frac{h_a}{2})^2) \\
          &+ 2 \cdot (\frac{1}{2} m_b r_b^2 + m_b(\frac{r_b}{2})^2)
\end{align}

Position 2 wird analog berechnet, allerdings wird vorab noch der Abstand der Mittelpunkte von
ausgestrecktem Arm und Bein zur Drehachse bestimmt:

\begin{align}
  \increment_a = \sqrt{(r_t + r_a)^2 + \frac{h_a^2}{4}} \\
  \increment_b = \sqrt{(r_t + r_b)^2 + \frac{h_b^2}{4}}
\end{align}

\begin{align}
  I_{ges} &= I_T + I_K + I_A + I_B - I_D\\
          &= \frac{1}{2} m_t r_t^2 + \frac{1}{2} m_k r_k^2 + (\frac{1}{2} m_a r_a^2 + m_a (r_a + r_t)^2) \\
          &+ (\frac{1}{2} m_b r_b^2 + m_b (\frac{r_t}{2})^2) + (m(\frac{r_b^2}{4} + \frac{h_b^2}{12}) + m_b\increment_2^2) \\
          &+ (m(\frac{r_a^2}{4} + \frac{h_a^2}{12}) + m_a\increment_1^2)
\end{align}

Somit ergeben sich folgende Trägheitsmomente:

\begin{align}
  I_{ges,pos1} = 0.001117 \,\, \symup{kg m^2} \\
  I_{ges,pos2} = 0.001251 \,\, \symup{kg m^2}
\end{align}

\subsubsection{Brechnung der Trägheitsmomente anhand der experimentellen Daten}

Wie zuvor kann nun das Trägheitsmoment auch anhand der experimentell ermittelten Daten (Tabelle 7) bestimmt
werden:

\begin{align}
  T_{pos1} &= (1.1284 \pm 0.0029) \symup{s} \\
  T_{pos2} &= (1.4096 \pm 0.0032) \symup{s}
\end{align}

Der angegebenen Fehler für die Trägheitsmomente errechnet sich nach \eqref{eqn:sigma}.
Somit ergeben sich für die Trägheitsmomente folgende Werte:

\begin{align}
  I_{pos1} &= (0.00083 \pm 0.00007) \symup{kg m^2} \\
  I_{pos2} &= (0.00136 \pm 0.00007) \symup{kg m^2}
\end{align}

\begin{table}
  \centering
  \caption{Schwingungsdauer der beiden Positionen der Holzpuppe}
  \begin{tabular}{c c | c c}
    \toprule $5T_{pos1} [s]$ & $T_{pos1} [s]$ & $5T_{pos2} [s]$ & $T_{pos2} [s]$ \\
    \midrule
    5.63 & 1.126 & 6.07 & 1.394 \\
    5.62 & 1.124 & 7.06 & 1.412 \\
    5.59 & 1.118 & 7.06 & 1.412 \\
    5.63 & 1.126 & 7.04 & 1.408 \\
    5.60 & 1.120 & 7.10 & 1.420 \\
    5.72 & 1.144 & 7.03 & 1.406 \\
    5.62 & 1.124 & 7.13 & 1.426 \\
    5.63 & 1.126 & 7.00 & 1.400 \\
    5.72 & 1.144 & 7.09 & 1.418 \\
    5.66 & 1.132 & 7.00 & 1.400 \\
    \bottomrule
  \end{tabular}
\end{table}
