\begin{table}
  \centering
  \caption{Kraftmessung des Eigenträgheitsmoments}
  \begin{tabular}{c c}
    \toprule
    $Grad$ & $Kraft\, \, in\, \,N$ \\
    \midrule
     45 & 0.22\\
     90 & 0.29\\
     135 & 0.40\\
     180 & 0.50\\
     225 & 0.60\\
     250 & 0.62\\
     270 & 0.63\\
     300 & 0.74\\
     315 & 0.74\\
     360 & 0.78\\
    \bottomrule
  \end{tabular}
\end{table}

%Stellung1 der Puppe
\begin{table}
  \centering
  \caption{Stellung 1 der Puppe}
  \begin{tabular}{c c}
    \toprule
    $Messung$ & $T\, \, in \, \,s$ \\
    \midrule
     1 & 1.60\\
     2 & 1.30\\
     3 & 1.38\\
     4 & 1.49\\
     5 & 1.46\\
    \bottomrule
  \end{tabular}
\end{table}

%Stellung2 der Puppe
\begin{table}
  \centering
  \caption{Stellung 2 der Puppe}
  \begin{tabular}{c c}
    \toprule
    $Messung$ & $T\, \, in \, \,s$ \\
    \midrule
     1 & 2.12\\
     2 & 2.00\\
     3 & 2.12\\
     4 & 2.00\\
     5 & 1.92\\
    \bottomrule
  \end{tabular}
\end{table}
