\subsection{Diskussion}
Die Messscheibe für die Winkelauslenkungen war nicht fest an der Apparatur befestigt.
Daher verschob sie sich leicht. Dies führt zu Ungenauigkeiten bei der Auslenkung.
Hinzu kommen die Fehler, die beim Ablesen geschehen.
Bei der Holzpuppe kam es vorallem zu Messungenauigkeiten,
da sie bei der Schwingung ihre Position nicht beibehalten konnte.
 Eine Messung mit angelegten Armen war nicht möglich,
da diese fest in waagerechter Stellung geklebt waren.
Die Ergebnisse für das Trägheitsmoment der Puppe mit ausgestreckten Armen und Beinen
stimmen nicht überein. Dies kann an den oben genannten Fehlern liegen.
