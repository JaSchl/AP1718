
\section{Auswertung}
\subsection{Bestimmung der Winkelrichtgröße $D$ und des Eigenträgheitsmoments $I_D$}

Der Kraftmesser hat zum Mittelpunkt einen Abstand von $\SI{20}{\centi\meter}$.
Mit diesem Radius $r$ wird nun die Winkelrichtgröße $D$ bestimmt.
Durch die Formel:
\begin{equation}
  D = \frac{F\cdot r}{\varphi}
\end{equation}
ergibt sich:
\begin{equation*}
 D = (0,032 \pm 0,007)\, \mathrm{Nm}.
\end{equation*}
Die Daten für diese Berechnung werden aus Tabelle \ref{tab:data2} entnommen.
Die Werte für $\varphi$ aus Tabelle \ref{tab:phi}

\begin{table}[H]
  \centering
  \caption{Kraftmessung des Eigenträgheitsmoments}
  \label{tab:phi}
  \begin{tabular}{c c c c}
    \toprule
    $F[N]$ & $\varphi[°]$ & $\varphi[rad]$ & $D[Nm]$\\
    \midrule
     0.22 & 45 & $\frac{\pi}{4}$ & 0,056 \\
     0.29 & 90 & $\frac{1}{2} \cdot \pi$ & 0,037 \\
     0.40 & 135 & $\frac{3}{4} \cdot \pi$ & 0,034 \\
     0.50 & 180 & $\pi$ & 0,032 \\
     0.60 & 225 & $\frac{5}{4} \cdot \pi$ & 0,031 \\
     0.62 & 250 & $\frac{25}{18} \cdot \pi$ & 0,028 \\
     0.63 & 270 & $\frac{3}{2} \cdot \pi$ & 0,027 \\
     0.74 & 300 & $\frac{5}{3} \cdot \pi$ & 0,028 \\
     0.74 & 315 & $\frac{7}{4} \cdot \pi$ & 0,027\\
     0.78 & 360 & $2 \cdot \pi$ & 0,025\\
    \bottomrule
  \end{tabular}
\end{table}

\begin{figure}[H]
 \centering
 \includegraphics[width=\textwidth]{plot1.pdf}
 \caption{Die Quadrate der Schwingungsdauer gegenüber den Abstandsquadraten}
 \label{fig:2}
\end{figure}

Die lineare Regression wird mittels Python durchgeführt. Für die Gerade
\begin{align}
 T^2 &= m \cdot a^2 + n
\end{align}
ergibt sich die Steigung $m = (784 \pm 26) \frac{s^2}{m^2}$
und der y-Achsenabschnitt $n = (4.3 \pm 1.1) s^2$.
\begin{equation*}
  m= \frac{8\pi^2m_{zyl}}{D}
\end{equation*}
\begin{equation}
  n= \frac{4\pi^2}{D}\cdot (I_D + I_{zyl,s})
  \label{eqn:eig}
\end{equation}

\begin{table}[H]
 \centering
 \caption{Schwingungsdauern bei jeweiligen Abständen.}
 \label{tab:data2}
 \begin{tabular}{c c c c }
   \toprule $a/ m$ & $a^2/ m^2$ & $T / s$ & $T^2/ s^2$ \\
   \midrule
   0.03 & 0.0009 & 2.06 &  4.2436\\
   0.06 & 0.0036 & 2.29 &  5.2441\\
   0.09 & 0.0081 & 3.21 &  10.3041\\
   0.12 & 0.0144 & 4.16 &  17.3056\\
   0.15 & 0.0225 & 4.81 &  23.1361\\
   0.18 & 0.0324 & 5.41 &  29.2681\\
   0.21 & 0.0441 & 6.38 &  40.7044\\
   0.24 & 0.0579 & 7.16 &  51.2656\\
   0.27 & 0.0729 & 7.52 &  56.5504\\
   0.29 & 0.0841 & 8.49 &  72.0801\\
   \bottomrule
 \end{tabular}
\end{table}
\newpage
Mit Formel (\ref{eqn:eig}) folgt nun die Formel:
\begin{equation}
  I_D = \frac{D}{4 \pi ^2}\cdot n - I_{Zyl}
\end{equation}
Die Formel für $I_{Zyl}$ ist in Formel (\ref{eqn:4}) zu finden.
Es ergibt sich:
\begin{equation*}
  I_{Zyl} = 1,14738 \cdot 10^{-5} \mathrm{kg\, m^2}
\end{equation*}
Dieses Trägheitsmoment ist nicht fehlerbehaftet, da der gemessene Radius $r$,
die Masse $m$ und die länge $h$ nicht fehlerbehaftet sind.
Somit ergibt sich nun das Eigenträgheitsmoment:
\begin{equation*}
 I_D = (0,0035\pm 0,0012) \symup{kg m^2} .
\end{equation*}
Der Fehler wurde mit Formel
\begin{equation}
  \Delta I_D = \sqrt{\left(\frac{D}{4\pi ^2}\cdot \Delta n \right)^2+
  \left(\frac{n}{4\pi ^2}\cdot \Delta D\right)^2}
\end{equation}
bestimmt.
\subsection{Bestimmung der Trägheitsmomente zwei unterschiedlicher Körper}
\subsubsection{Trägheitsmoment eines Zylinders}
Zu Beginn der Messung werden von dem ausgewählten Zylinder der Radius $r$,
die Höhe $h$ und die Masse $m$ gemessen:
\begin{align*}
r_Z &= 0.04 \, \symup{m} \\ %Fehler?
h_Z &= 0.14  \, \symup{m} \\
m_Z &= 0.8995 \, \symup{kg}
\end{align*}
Anhand dieser Werte wird das theoretische Trägheitsmoment bestimmt:
\begin{equation*}
I_{Zylinder,theo} = \frac{1}{2} m_Z \cdot r_Z^2 = 0.000720 \,\, \symup{kg m^2}
\end{equation*}
\begin{table}[H]
  \centering
  \label{tab:zyl}
  \caption{Schwingungsdauer eines Zylinders}
  \begin{tabular}{c }
    \toprule
     $T/s$ \\
    \midrule
      1.41 \\
      1.35 \\
      1.32 \\
      1.41 \\
     1.40 \\
    \bottomrule
  \end{tabular}
\end{table}
Für die Schwingungsdauer ergibt sich durch Mitteln,
der in Tabelle (3) aufgelisteten Werte
\begin{equation*}
\bar{T}_{Zylinder} = (1.378 \pm 0.018) \symup{s}.
\end{equation*}
und für das Trägheitsmoment des Zylinder durch Formel
\begin{equation}
  I = \frac{T^2\cdot D}{4\pi ^2}-I_D
  \label{eqn:träg}
\end{equation}
\begin{equation*}
I_{Zylinder,exp} = (-0,0020 \pm 0,0012)  \cdot 10^{-5}\symup{kg m^2} %Fehler und Wert komisch, D=?
\end{equation*}
Die Fehlerrechnung wurde mit Formel
\begin{equation}
  \Delta I = \sqrt{\left(\frac{T^2}{4\pi^2}\cdot \Delta D\right)^2
  +\left(\frac{DT}{2\pi^2}\cdot \Delta T\right)^2 + \Delta I_D^2}
  \label{eqn:fehler}
\end{equation}
durchgeführt.
\subsubsection{Trägheitsmoment einer Kugel}

Zunächst wird die Kugel ausgemessen und gewogen. Dardurch ergibt sich Die Masse $m$
und der Radius $r$ :
\begin{align*}
r_K &= 0,695 \, \symup{m} \\
m_K &= 0,8125 \, \symup{kg}
\end{align*}
Anhand dieser Werte kann nun das Trägheitsmoment $I_{Kugel, theo}$ bestimmt werden.
Dies geschieht mit Formel (\ref{eqn:3}):
\begin{equation*}
I_{Kugel,theo}= 0,0015698 \, \, \symup{kg \, m^2}
\end{equation*}
Der experimentelle Wert für das Trägheitsmoment ermittelt sich aus Formel (\ref{eqn:träg}).
Der Wert für die Schwingungsdauer $T$ wird mit Tabelle (\ref{tab:schw}) bestimmt.
Der Körper wird immer um einen Winkel von $ \frac{4\pi}{3}$ ausgelenkt.

\begin{table}
\centering
\caption{Schwingungsdauer der Kugel}
\label{tab:schw}
\begin{tabular}{c}
\toprule
$T/s$ \\
\midrule
1,63\\
1,55\\
1,55\\
1,53\\
1,58\\
\bottomrule
\end{tabular}
\end{table}
Für den Mittelwert ergibt sich:
\begin{equation*}
\bar{T}= (1{,}568 \pm 0{,}017) \, \mathrm{s} .
\end{equation*}
Der experimentelle Wert für das Trägheitsmoment lautet somit:
\begin{equation*}
I_{Kugel,exp}= (-0,0015 \pm 0,0013) \cdot 10^{-3} \, \mathrm{kg \, m^2} %bezug Fehlerrechnung
\end{equation*}
Der Fehler werden ebenfalls mit Formel (\ref{eqn:fehler}) bestimmt.
\subsection{Bestimmung des Trägheitsmoments einer Puppe}

\subsection{theoretische Bestimmung}

Eine Holzpuppe hat ein Gewicht von
\begin{equation*}
m= 0,03425 \, \mathrm{kg}
\end{equation*}
und wird als Zylinder genährt.
Dafür werden die Körperteile einzeln betrachtet.
Hier bei ist die Höhe bzw. die Länge  $h$ und der Radius $r$. \\

\emph{Kopf}:
\begin{align*}
h &= 0,071 \, \mathrm{m} & r &= (1,6725 \pm 0,051) \cdot 10^{-2} \, \mathrm{m}
\end{align*}
\emph{Rumpf}:
\begin{align*}
h &= 0,122 \, \mathrm{m} & r &= (2,292 \pm 0,406) \cdot 10^{-2} \, \mathrm{m}
\end{align*}
\emph{Arm}:
\begin{align*}
h &= 0,179 \, \mathrm{m} & r &= (0,871 \pm 0,051) \cdot 10^{-2} \, \mathrm{m}
\end{align*}
\emph{Bein}:
\begin{align*}
h &= 0,198 \, \mathrm{m} & r &= (0,932 \pm 0,145) \cdot 10^{-2} \, \mathrm{m}
\end{align*}
Die Höhen und die Radien mit dem Fehler ergeben sich aus den gemessenen Werten.
Die Radien sind in Tabelle (\ref{tab:rad}) zu finden.
\begin{table}[H]
  \centering
  \caption{Radien der Einzelnen Körperteile}
  \label{tab:rad}
  \begin{tabular}{c c c c}
\toprule
$r_{Kopf}/cm$ & $r_{Arm}/cm$ & $r_{Rumf}/cm$ & $r_{Bein}/cm$ \\
\midrule
1,40 & 0,90 & 2,75 & 1,25 \\
1,65 & 0,96 & 2,25 & 0,73 \\
1,84 & 0,73 & 1,77 & 1,10 \\
1,80 & 0,90 & 2,40 & 0,65 \\
\bottomrule
\end{tabular}
\end{table}
Um das Trägheitsmoment theoretisch bestimmen zu können, müssen die Massen der einzelnen Körperteile bestimmt werden.
Dies geschieht mit der Formel
\begin{equation*}
m_{teil}= \frac{V_{teil}}{V_{ges}}\cdot m_{ges}
\end{equation*}

Die dafür benötigten Volumen ergeben sich aus der Formel
\begin{equation*}
V_{Zylinder}= \pi \cdot r^2 \cdot h
\end{equation*}
\begin{align*}
V_{Kopf} &= (0,6239366 \pm 0,0000038) \, \mathrm{m^3} \\
V_{Rumpf} &= (2,0134411 \pm 0,0000713) \, \mathrm{m^3} \\
V_{Arme} &= (0,4266180 \pm 0,0000049) \, \mathrm{m^3} \\
V_{Beine} &= (0,5403148 \pm 0,0000168) \, \mathrm{m^3} \\
V_{Gesamt} &= (4,571233 \pm 0,000119) \, \mathrm{m^3}
\end{align*}
Die Fehler wird mit der Formel
\begin{equation*}
  \Delta V_{Zyl.}= (2\cdot \pi \cdot r \cdot h \cdot \Delta r)
\end{equation*}
berechnet.
Somit ergeben sich nun die einzelnen Massen:
\begin{align*}
m_{Kopf} &= (4,67485 \pm 0,00012) \cdot 10^{-3} \, \mathrm{kg}\\
m_{Rumpf} &= (15,08572 \pm 0,00066)\cdot 10^{-3} \, \mathrm{kg}\\
m_{Arme} &= (3,19644) \pm 0,00009) \cdot 10^{-3} \, \mathrm{kg}\\
m_{Beine} &= (4,04831 \pm 0,00126) \cdot 10^{-3} \, \mathrm{kg}
\end{align*}
Die Massenangaben sind jeweils für beide Arme und Beine angegeben.
Die Fehler werden mit Formel
\begin{equation*}
  \Delta m_{teil} = \sqrt{\left(\frac{m_{ges.}}{V_{ges.}}\cdot \Delta V_{teil}\right)^2\, + \,
  \left(-\frac{V_{teil} \cdot m_{ges.}}{V_{ges.}^2}\cdot \Delta V_{ges}\right)^2}
\end{equation*}
 bestimmt.
 Das Theoretische Trägheitsmoment ergibt sich aus der Formel:
 \begin{equation}
   \sum I_i = I_{Rumpf} +I_{Kopf} + 2\cdot I_{Arm} + 2\cdot I_{Bein}
 \end{equation}
Es ist zu Beachten, dass bei Position 1 für die Arme und bei Position 2 für die Arme und Beine
der Satz von Steiner Anzuwenden ist.
Die Formel hierfür lautet:
\begin{equation}
  I_{Arm/Bein} = m \left( \frac{r^2}{4} + \frac{\left( r_{rumpf}+{1/2 \cdot l_{Arm/Bein}}\right)}{12}\right)
\end{equation}
Die einzelnen Trägheitsmomente,
\begin{align*}
  I_{Kopf} &= (0,002477 \pm 0,000009) \, \mathrm{kg\, m^2}\\
  I_{Rumf} &= (0,00781 \pm 0,00009) \, \mathrm{kg\, m^2}\\
  I_{Arm} &= (3,69 \pm 0,12) \cdot 10^{-5}\, \mathrm{kg\, m^2}\\
  I_{Bein,pos.1} &= (1,8 \pm 0,5) \cdot 10^{-7}\, \mathrm{kg\, m^2}\\
  I_{Bein,pos.2} &= (5.06 \pm 0,20) \cdot 10^{-5}\, \mathrm{kg\, m^2}
\end{align*}
summieren sich zu den Trägheitsmomenten der beiden Positionen.
Sie lauten:
\begin{equation*}
  I_{p1,theo} = (0,01032408 \pm 0,0001025) \, \mathrm{kg \, m^2}
\end{equation*}
\begin{equation*}
  I_{p2,theo} = (0,0103745 \pm 0,0001022 ) \, \mathrm{kg \, m^2}
\end{equation*}

\subsection{experimentelle Bestimmung}
Für eine stehende Holzpuppe mit ausgestreckten Armen ergibt sich mit den
Werten aus Tabelle (\ref{tab:4}) eine Schwingungsdauer von
\begin{equation*}
\bar{T}= (1,446 \pm 0,051) \, \mathrm{s}
\end{equation*}
und somit das Trägheitsmoment,
\begin{equation*}
  I_{p1,exp} = (0,066 \pm 0,002) \mathrm{kg \, m^2}
\end{equation*}
 das nach Formel (\ref{eqn:träg}) bestimmt wird.
 Der Fehler wird mit Formel(\ref{eqn:fehler}) bestimmt.

 \begin{table}[H]
   \centering
   \caption{Stellung 1 der Puppe}
   \label{tab:4}
   \begin{tabular}{c c}
     \toprule
      $T/s$ \\
     \midrule
       1.60\\
      1.30\\
      1.38\\
      1.49\\
      1.46\\
     \bottomrule
   \end{tabular}
 \end{table}

Für diese Puppe mit ausgestreckten Armen und angewinkelten Beinen,
ergibt sich die Schwingungsdauer
\begin{equation*}
\bar{T}= (2,032 \pm 0,039) \, \mathrm{s}
\end{equation*}
und das Trägheitsmoment
\begin{equation*}
  I_{p2,exp}= (0,067 \pm 0,003) \mathrm{Kg \, m^2} ,
\end{equation*}
das mit Formel (\ref{eqn:träg}) berechnet wurde.
Die Werte für die Schwingungsdauer wurden Tabelle \ref{tab:5} entnommen.

\begin{table}
  \centering
  \caption{Stellung 2 der Puppe}
  \label{tab:5}
  \begin{tabular}{c c}
    \toprule
    $T\, \, in \, \,s$ \\
    \midrule
     2.12\\
     2.00\\
     2.12\\
     2.00\\
     1.92\\
    \bottomrule
  \end{tabular}
\end{table}
