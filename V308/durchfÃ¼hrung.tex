

\section{Durchführung}
\subsection{Magnetfeld von Spulen}
Im ersten Teil des Versuchs werden die Magnetfelder einer langen und einer kurzen Spule gemessen.
Hierfür wird eine longitudinale Sonde so eingestellt,
dass sie direkt durch die Mitte der Spulen geschoben werden kann.
Bei der kurzen Spule wird ein Strom von 1 Ampere  angelegt,
bei der langen ein Strom von 1,1 Ampere.
Es werden Messwerte außerhalb und innerhalb der Spule
in einem Abstand von $0,5 -1\, \mathrm{cm}$ aufgenommen.
\subsection{Magnetfeld einer Helmholtzspule}
Das Spulenpaar wird in Reihe geschaltet und an das Netzgerät angeschlossen.
Es wird ein Strom von $1,2\, A$ angelegt.
Die Feldstärke wird bei drei unterschiedlichen Abständen zwischen den Spulen gemessen.
Dafür werden Abstände von 8, 11 und 14\, $\mathrm{cm}$ gewählt.
Mit einer transversalen Hall-Sonde wird das Magnetfeld jeweils innerhalb und außerhalb des Spulenpaars gemessen.
%Der Aufbau ist in Abbildung \ref{fig:2} zu sehen.
\subsection{Die Hysteresekurve}
Für die Messung der Hysteresekurve wird eine Ringspule mit Eisenkern benötigt.
In diesem Teil des Versuchs wird das Magnetfeld als Funktion des Spulenstroms gemessen.
Dies geschieht ebenfalls mit einer longitudinalen Hall-Sonde.
%Der Aufbau ist in Abbildung \ref{:fig:3} zu sehen.
Der Strom wird von 0 auf 10 Ampere hochgedreht. So entsteht die Neukurve.
Um den Rest der Hysteresekurve zu bekommen,
wird der Strom wieder auf Null herunter gedreht.
An diesem Punkt ist eine Umpolung erforderlich, um die Negativen Werte der Kurve zu erzeugen.
Es wird wieder auf 10 Ampere hochgedreht und wieder auf Null Ampere heruntergedreht.
Es ist wieder eine Umpolung erforderlich.
Der Strom wird nun noch einmal auf 10 Ampere hochgedreht.
Es zu beachten, dass die Ringspule vorher entmagnetisiert werden muss.
Außerdem ist es sehr wichtig den Strom bei der Durchführung nicht wieder hoch oder runter zu drehen.
Dies würde die Messung verfälschen, da auf den anderen Ast der Hysteresekurve gewechselt wird.
